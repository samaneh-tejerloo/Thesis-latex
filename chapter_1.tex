\chapter{معرفی پژوهش}
\section{مقدمه}
بیوانفورماتیک سعی بر پاسخ به مسائل و پرسش‌های زیست شناسی به کمک ابزار‌های محاسباتی و مدل‌سازی‌های آماری دارد. یافتن پاسخ مناسب برای هر یک از این مسائل می‌تواند تاثیر به سزایی در فهم بیشتر ما از عملکرد‌های زیستی داشته باشد. در این پژوهش، ما بر روی پروتئين‌‌ها تمرکز کرده‌ایم و هدف شناسایی مجموعه‌های پروتئینی، به کمک شبکه‌ تعاملات پروتئین-پروتئین می‌باشد.

پروتئین‌ها مولکول‌های بزرگ و پیچیده‌ای هستند که وظایف زیستی حیاتی‌ای مانند نقش ساختاری و حمایتی از سلول‌ها، عملکرد پادتنی، نقش‌های پیام رسانی و یا نقش آنزیمی را عهده دار هستند. 
بسیاری از فرآیند‌های زیستی به وسیله مجموعه‌ای از پروتئین‌ها انجام می‌گیرد که ماژول‌های عملکردی نامیده می‌شوند. شناخت هر چه بهتر این مجموعه‌های پروتئینی به درک بهتر ما از فرايند‌های زیستی و همچین درک نقش پروتئین‌های کمتر شناخته شده کمک شایانی می‌کند. تشخیص این ماژول‌ها به کمک روش‌های آزمایشگاهی سخت و هزینه‌بر است از این روی تلاش‌های بسیاری (مقاله سایت بزنیم) در جهت ارائه روش‌های محاسباتی کارا به منظور شناسایی این مجموعه‌های پروتئینی انجام شده است.
در ادامه این بخش به بیان بهتر مسئله پیش‌رو و همچنین مفاهیم بنیادی مورد نیاز می‌پردازیم.




\section{بیان مسئله}
با پایان یافتن پروژه توالی‌یابی ژنوم انسان و ورود به دوره‌ی پسا‌ژنی\LTRfootnote{\lr{Postgenomic era}}، پژوهش‌های پروتئومیک\LTRfootnote{\lr{Proteomics}} به یکی از مهم‌ترین و فعال‌ترین حوزه‌های علوم زیستی تبدیل شده‌اند. پروتئومیک به مطالعه جامع ویژگی‌های پروتئین‌ها با هدف توصیف ساختار، عملکرد و سازوکارهای کنترلی سیستم‌های زیستی می‌پردازد. پروتئین‌ها عموماً به صورت منفرد عمل نمی‌کنند، بلکه از طریق تعامل با یکدیگر مجموعه‌هایی را تشکیل می‌دهند که انجام بسیاری از وظایف حیاتی سلولی را امکان‌پذیر می‌سازند. تعاملات پروتئین–پروتئین\LTRfootnote{\lr{Protein–protein interactions (PPI)}} نقشی اساسی در فرآیندهایی نظیر تکثیر ماده ژنتیکی\LTRfootnote{\lr{Gene substance copy}}، کنترل بیان ژن\LTRfootnote{\lr{Gene expression control}}، انتقال سیگنال‌های سلولی\LTRfootnote{\lr{Cellular signal transduction}} و مرگ برنامه‌ریزی‌شده سلولی\LTRfootnote{\lr{Cell apoptosis}} ایفا می‌کنند. از این‌رو، تحلیل شبکه‌های \lr{PPI} برای درک عمیق‌تر سازماندهی و عملکرد سلولی امری ضروری محسوب می‌شود \cite{survey}.

زیست‌انفورماتیک\LTRfootnote{\lr{Bioinformatic}} به عنوان یک حوزه میان‌رشته‌ای، با تلفیق دانش زیست‌شناسی، علوم کامپیوتر، ریاضیات و آمار، به ذخیره‌سازی، مدیریت و تحلیل داده‌های زیستی می‌پردازد. این علم با بهره‌گیری از ابزارها و فناوری‌های محاسباتی، داده‌های مرتبط با توالی‌های دی‌اِن‌اِی\LTRfootnote{\lr{DNA}}، آر‌اِن‌اِی\LTRfootnote{\lr{RNA}} و پروتئین‌ها را پردازش و تفسیر می‌کند و با توجه به حجم عظیم داده‌ها و اهمیت استخراج دانش کاربردی از آن‌ها، جایگاهی کلیدی در پژوهش‌های نوین زیستی یافته است \cite{BioinformaticReview}.

مطالعات زیستی نشان می‌دهد که یک مجموعه پروتئینی\LTRfootnote{\lr{Protein complex}} در شبکه‌های \lr{PPI} به صورت یک ساختار مولکولی منسجم تعریف می‌شود که از پروتئین‌هایی با سازگاری عملکردی و ساختاری تشکیل شده است \cite{survey}. به بیان دیگر، پروتئین‌هایی که در شبکه \lr{PPI} با یکدیگر تعامل دارند، غالباً از منظر کارکردهای زیستی نیز شباهت‌های معناداری از خود نشان می‌دهند. بر این اساس، زیرشبکه‌های به‌هم‌پیوسته و با تراکم بالای پروتئین‌ها می‌توانند به عنوان ماژول‌های عملکردی\LTRfootnote{\lr{Functional module}} یا مجموعه‌های پروتئینی در نظر گرفته شوند که در انجام فرآیندهای زیستی خاص نقش دارند \cite{functional}. شناسایی این ساختارها علاوه بر امکان بررسی تعامل میان مجموعه‌های پروتئینی مختلف، می‌تواند به کشف مجموعه‌های پروتئینی ناشناخته نیز منجر شود \cite{survey}.

یکی از رویکردهای کارآمد برای مطالعه شبکه‌های \lr{PPI}، مدل‌سازی و تحلیل آن‌ها از منظر نظریه گراف و شبکه‌های پیچیده است. با اتخاذ این دیدگاه و با توجه به مقیاس بزرگ داده‌های زیستی، یکی از چالش‌های اساسی در دوره پسا‌ژنی، طراحی الگوریتم‌های بهینه و مقیاس‌پذیر به منظور شناسایی مؤثر ماژول‌های عملکردی و مجموعه‌های پروتئینی زیستی است \cite{spectral}. از آنجایی که پروتئین‌های یک مجموعه پروتئینی در شبکه \lr{PPI} دارای تعاملات متراکم و فراوانی با یکدیگر هستند، این نواحی متراکم در شبکه را می‌توان به عنوان مجموعه‌های پروتئینی احتمالی در نظر گرفت. در نتیجه، مسئله شناسایی مجموعه‌های پروتئینی شباهت زیادی به مسئله خوشه‌بندی در شبکه‌های پیچیده دارد \cite{computational} و می‌توان آن را به صورت یک مسئله خوشه‌بندی گرافی مدل‌سازی کرد.

بخش عمده‌ای از پژوهش‌های پیشین در این حوزه صرفاً بر پایه اطلاعات ساختاری شبکه‌های \lr{PPI} ارائه شده‌اند \cite{automated}, \cite{detecting}، در حالی که امروزه داده‌های غنی و متنوعی برای توصیف ویژگی‌های پروتئین‌ها در دسترس است. به عنوان نمونه، پایگاه داده هستی شناسی ژن\LTRfootnote{\lr{Gene Ontology}} اطلاعات جامعی درباره ژن‌ها و پروتئین‌ها از منظر فرآیندهای زیستی، عملکردهای مولکولی و مولفه‌های سلولی فراهم می‌کند و به طور گسترده مورد استفاده پژوهشگران قرار گرفته است \cite{gene}. افزون بر این، از آنجا که پروتئین‌ها محصولات بیان ژن هستند، برخی مطالعات از داده‌های بیان ژنی مرتبط با هر پروتئین نیز به منظور توصیف دقیق‌تر آن‌ها در شبکه‌های \lr{PPI} بهره برده‌اند \cite{integration}. ترکیب این اطلاعات تکمیلی با ساختار شبکه \lr{PPI} می‌تواند به شناسایی دقیق‌تر و مؤثرتر مجموعه‌های پروتئینی منجر شود.

برای بهره‌برداری مناسب از این اطلاعات، نیاز به الگوریتم‌های خوشه‌بندی گرافی وجود دارد که بتوانند به صورت هم‌زمان ساختار شبکه و ویژگی‌های گره‌ها را در فرآیند خوشه‌بندی در نظر بگیرند. به این دسته از روش‌ها، خوشه‌بندی گراف‌های دارای ویژگی\LTRfootnote{\lr{Attributed graph clustering}} گفته می‌شود \cite{clustering}. از سوی دیگر، با توجه به این‌که یک پروتئین می‌تواند در چندین فرآیند زیستی مختلف مشارکت داشته باشد، بسیاری از مجموعه‌های پروتئینی دارای هم‌پوشانی هستند؛ بنابراین الگوریتم مورد نظر باید قادر به شناسایی مجموعه‌های پروتئینی هم‌پوشان نیز باشد.

اگرچه بسیاری از پژوهش‌های پیشین تنها با تکیه بر ساختار شبکه‌های \lr{PPI} و بدون در نظر گرفتن ویژگی‌های شناخته‌شده پروتئین‌ها، الگوریتم‌های کلاسیکی را برای شناسایی ماژول‌های عملکردی ارائه کرده‌اند، اما با پیشرفت‌های اخیر در حوزه هوش مصنوعی و الگوریتم‌های یادگیری ماشین، به‌ویژه یادگیری عمیق، ضرورت بررسی توانمندی شبکه‌های عصبی، به‌خصوص شبکه‌های عصبی گرافی، در این زمینه بیش از پیش احساس می‌شود. هرچند در برخی مطالعات، ترکیبی از روش‌های کلاسیک و شبکه‌های عصبی مورد استفاده قرار گرفته است، اما تاکنون یک رویکرد کاملاً مبتنی بر یادگیری عمیق و به صورت یکپارچه\LTRfootnote{\lr{End-to-end}} به طور جدی مورد توجه قرار نگرفته است.

در مجموع، هدف این پژوهش ارائه یک الگوریتم گراف‌محور برای کشف مجموعه‌های پروتئینی و ماژول‌های عملکردی هم‌پوشان در شبکه‌های \lr{PPI}، با بهره‌گیری هم‌زمان از ساختار شبکه و داده‌های تکمیلی استخراج‌شده از پایگاه‌های داده زیستی، در قالب چارچوب خوشه‌بندی گراف‌های دارای گره‌های ویژگی‌دار است. در ادامه، اهمیت موضوع، پرسش‌های اصلی پژوهش و روش پیشنهادی به صورت خلاصه مورد بررسی قرار خواهند گرفت.

\section{اهمیت و ضرورت انجام پژوهش}

در حال حاضر زیست شناسان توجه خود را از مطالعه ساختار و عملکرد انفرادی پروتئین‌ها به سمت مطالعه ساختاری و عملکردی ‌‌مجموعه‌های پروتئینی تغییر داده‌اند و مولکول‌های پروتئین‌ها را درون یک شبکه زیستی کلی بررسی می‌کنند \cite{farutin2006edge}.
دلیل این موضوع این است که بررسی یک پروتئین باید در ارتباط با سایر پروتئین‌ها و مجموعه‌های پروتئینی‌ که به آن‌ها تعلق دارد، انجام شود.
از سوی دیگر، جهش‌های موجود در دی‌اِن‌اِی می‌توانند نحوه برهم‌کنش پروتئین‌ها را در یک مجموعه پروتئینی تغییر دهند و به این ترتیب باعث تغییر در عملکرد و رفتار آن مجموعه شوند. این تغییرات نقش مهمی در بررسی فرایند توسعه داروها و همچنین شناخت علل بروز بیماری‌ها دارند \cite{altaf2006development, zaki2018improving}.
به عنوان مثال پژوهش‌های \cite{farutin2006edge, macropol2009rrw}، نشان داده‌اند که برخی از بیماری‌های ژنتیکی به وسیله پروتئین‌های با برهم‌کنش‌های عملکردی مشابه به وجود می‌آیند. همچنین مجموعه‌های پروتئینی به واسطه ارتباط‌‌شان با مسیر‌های زیستی\LTRfootnote{Pathway}، برای فهم بهتر نحوه توزیع، جذب، متابولیسم و دفع دارو ضروری هستند. از این روی شناسایی مجموعه‌های پروتئینی برای کشف و توسعه دارو‌ها اهمیت زیادی دارند.
با وجود اهمیت کشف مجموعه‌های پروتئینی، چالش اصلی زمان‌بر و هزینه‌بر بودن فرآیند کشف آن‌ها در آزمایشگاه است که سبب توجه بیشتر محققان به روش‌های محاسباتی جهت کشف ماژول‌های عملکردی شده است \cite{adappi}.

\section{پرسش‌های پژوهش}
% TODO:: COMPLETE THIS SECTION
بعد از تکمیل قسمت روش

\section{روش پژوهش}
% TODO:: COMPLETE THIS SECTION
بعدا تکمیل شود
\section{جمع‌بندی}
در این فصل به تعریف مسئله پژوهش و اهمیت آن پرداختیم. همچنین به صورت کلی روش پیشنهادی و سوالات پیش‌روی پژوهش را مورد بررسی قرار دادیم.

در ادامه، در فصل دوم به مبانی پایه مورد نیاز جهت فهم بهتر پژوهش اشاره خواهد شد. فصل سوم به معرفی پژوهش‌های پیشین که در زمینه شناسایی ماژول‌های عملکردی و مجموعه پروتئینی برپایه روش‌های محاسباتی هستند اختصاص دارد. فصل چهارم مربوط به روش شناسی و پژوهش است که در ابتدا داده‌های استفاده شده در این پژوهش سپس روش پیشنهادی خود را مطرح می‌کنیم. در فصل پنجم، به بررسی پیاده‌سازی روش پیشنهادی و نتایج حاصل بر اساس معیار‌های ارزیابی می‌پردازیم. همچنین عملکرد روش پیشنهادی را با دیگر روش‌های پیشین مقایسه کرده و برتری روش خود را شرح می‌دهیم. در نهایت، در فصل آخر، پژوهش حاضر را جمع‌بندی می‌کنیم و ایده‌هایی را برای ادامه‌ی مسیر این پژوهش مطرح می‌کنیم.