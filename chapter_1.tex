\chapter{معرفی پژوهش}
\section{مقدمه}
بیوانفورماتیک سعی بر پاسخ به مسائل و پرسش‌های زیست شناسی به کمک ابزار‌های محاسباتی و مدل‌سازی‌های آماری دارد. یافتن پاسخ مناسب برای هر یک از این مسائل می‌تواند تاثیر به سزایی در فهم بیشتر ما از عملکرد‌های زیستی داشته باشد. در این پژوهش، ما بر روی پروتئين‌‌ها تمرکز کرده‌ایم و هدف شناسایی مجموعه‌های پروتئینی، به کمک شبکه‌ تعاملات پروتئین-پروتئین می‌باشد.

پروتئین‌ها مولکول‌های بزرگ و پیچیده‌ای هستند که وظایف زیستی حیاتی‌ای مانند نقش ساختاری و حمایتی از سلول‌ها، عملکرد پادتنی، نقش‌های پیام رسانی و یا نقش آنزیمی را عهده دار هستند. 
بسیاری از فرآیند‌های زیستی به وسیله مجموعه‌ای از پروتئین‌ها انجام می‌گیرد که ماژول‌های عملکردی نامیده می‌شوند. شناخت هر چه بهتر این مجموعه‌های پروتئینی به درک بهتر ما از فرايند‌های زیستی و همچین درک نقش پروتئین‌های کمتر شناخته شده کمک شایانی می‌کند. تشخیص این ماژول‌ها به کمک روش‌های آزمایشگاهی سخت و هزینه‌بر است از این روی تلاش‌های بسیاری (مقاله سایت بزنیم) در جهت ارائه روش‌های محاسباتی کارا به منظور شناسایی این مجموعه‌های پروتئینی انجام شده است.
در ادامه این بخش به بیان بهتر مسئله پیش‌رو و همچنین مفاهیم بنیادی مورد نیاز می‌پردازیم.




\section{بیان مسئله\footnote{\lr{Problem statement}}}
بیوانفورماتیک\footnote{\lr{Bioinformatic}}  یک حوزه میان‌رشته‌ای است که با استفاده از علوم زیست‌شناسی، کامپیوتر، ریاضیات و آمار به ذخیره‌سازی و تحلیل داده‌های زیستی می‌پردازد. این علم با بهره‌گیری از فناوری‌های کامپیوتری، داده‌های مربوط به توالی‌های DNA، RNA و پروتئین‌ها را مدیریت و تفسیر می‌کند و به دلیل حجم بالای داده‌ها و اهمیت استخراج اطلاعات کاربردی، جایگاه ویژه‌ای دارد \cite{BioinformaticReview}. 

با پایان یافتن پروژه توالی‌یابی ژنوم انسان و ورود به دوره‌ی پسا‌ژنی\footnote{\lr{Postgenomic era}} ، تحقیقات پروتئومیک\footnote{\lr{Proteomics}}  به یکی از مهم‌ترین حوزه‌های علوم زیستی تبدیل شده است. پروتئومیک به مطالعه ویژگی‌های پروتئین‌ها برای توصیف ساختار، عملکرد و کنترل سیستم‌های زیستی می‌پردازد. پروتئین‌ها اغلب به تنهایی عمل نمی‌کنند، بلکه با هم تعامل دارند و برای انجام وظایف زیستی، به مولکول‌های بزرگ‌تری تبدیل می‌شوند. تعاملات پروتئین-پروتئین\footnote{\lr{Protein‌-protein interactions (PPI)}}  در فرآیندهای مهمی مانند تکثیر ژنتیکی\footnote{\lr{Gene substance copy}} ، کنترل بیان ژن\footnote{\lr{Gene expression control}} ، انتقال سیگنال‌های سلولی\footnote{\lr{Cellular signal transduction}}  و مرگ سلولی\footnote{\lr{Cell apoptosis}}  نقش کلیدی دارند. تحلیل شبکه‌های PPI برای درک بهتر سازماندهی و عملکرد سلولی ضروری است \cite{survey}. 

تحقیقات زیستی نشان می‌دهد که یک مجموعه پروتئینی\footnote{\lr{Protein complex}}  در شبکه‌های PPI یک ساختار مولکولی است که هم از نظر ویژگی و هم از نظر ساختاری از پروتئین‌های سازگار با هم تشکیل شده است\cite{survey}. به صورت شهودی نیز، در شبکه‌های PPI اگر دو پروتئین با هم تعامل داشته باشند، به احتمال بیشتری از نظر کارایی سلولی نیز شبیه‌ به یکدیگر هستند. از این رو پیدا کردن زیرشبکه‌های به هم متصل با تراکم بالا از پروتئین‌ها می‌تواند به عنوان ماژول‌های عملکردی\footnote{\lr{Functional module}}  و یا یک مجموعه پروتئینی در نظر گرفته شوند که در فرآیند‌های ویژه‌ای در سلول نقش دارند \cite{functional}. همینطور بر این اساس می‌توان تعامل و ارتباط بین مجموعه های پروتئینی مختلف را بررسی کرد و یا حتی مجموعه‌های پروتئینی ناشناخته‌ای را کشف کرد \cite{survey}.

یکی از بهترین روش‌ها برای مطالعه شبکه‌های PPI نگاه به آن‌ها از دید تحلیل شبکه‌های پیچیده و دید گرافی است. با این دید و به دلیل حجم عظیم داده‌های این شبکه‌ها، یکی از چالش‌های اساسی در دوره پسا‌ژنی، ارائه الگوریتم‌های بهینه به منظور شناسایی مؤثر ماژول‌های عملکردی زیستی و مجموعه‌های پروتئینی است \cite{spectral}. 
از آنجایی که پروتئین‌های یک مجموعه پروتئینی در شبکه PPI دارای تعاملات زیادی بین خودشان هستند و این موضوع باعث کارکرد مشابه آن‌ها می‌شود، در نتیجه نواحی پر تراکم در شبکه PPI را می‌توان به عنوان یک مجموعه پروتئینی احتمالی در نظر گرفت و شناسایی مجموعه‌های پروتئینی بسیار شبیه به پیدا کردن خوشه‌ها در یک شبکه پیچیده است \cite{computational}، از این رو می‌توان این مسئله را معادل خوشه‌بندی در گراف‌ها در نظر گرفت.

بیشتر‌ پژوهش‌های پیشین در این حوزه تنها از اطلاعات ساختاری شبکه‌های PPI \cite{automated}, \cite{detecting} (یه دو تا مقاله دیگه هم اضاقه بشه) استفاده کرده‌اند در حالی که امروزه ما به داده‌های غنی و مناسبی برای توصیف پروتئین‌ها دسترسی داریم. به عنوان مثال می‌توان به بانک اطلاعاتی GO\footnote{\lr{Gene Ontology}}  اشاره کرد که اطلاعاتی درباره ژن‌ها و پروتئین‌ها از دید‌های مختلف شامل فرآیند‌های زیستی، عملکرد مولکولی و مولفه‌های سلولی فراهم کرده و مورد بررسی و توصیف قرار می‌دهد \cite{gene}. همچنین از آنجایی که پروتئین‌ها محصولات ژنی هستند برخی از پژوهش‌ها از بیان ژنی مربوط به هر پروتئین نیز به منظور توصیف آن‌ها در شبکه PPI بهره‌ برده‌اند \cite{integration}. استفاده از این اطلاعات در کنار ساختار شبکه PPI می‌تواند منجر به تشخیص بهینه و موثرتر مجموعه‌های پروتئینی شود. برای استفاده مناسب از اطلاعات تکمیلی نیاز به الگوریتم‌های خوشه‌بندی گرافی داریم که به طور مناسب ویژگی‌های پروتئین‌ها را در کنار ساختار شبکه برای خوشه‌بندی مجموعه‌های پروتئینی در نظر گیرد، به این دسته از الگوریتم‌ها، خوشه‌بندی گراف‌های دارای ویژگی\footnote{\lr{Attributed graph clustering}}  می‌گویند \cite{clustering}. آز آنجایی که یک پروتئین می‌تواند در چندین فرآیند زیستی نقش داشته باشد، در نتیجه بسیاری از مجموعه‌‌های پروتئینی شامل پروتئین‌های مشترک ‌هستند. بنابراین الگوریتم پیشنهادی باید توانایی شناسایی مجموعه‌های همپوشان را نیز داشته باشد.

بسیاری از پژوهش‌های پیشین تنها با توجه به ساختار شبکه‌های PPI و بدون در نظر گرفتن ویژگی‌های شناخته شده پروتئین‌ها اقدام به ارائه الگوریتم‌های کلاسیک به منظور شناسایی ماژول‌های عملکردی کرده‌اند. با توجه به پیشرفت‌های اخیر در زمینه هوش مصنوعی و الگوریتم‌های یادگیری ماشین به ویژه الگوریتم‌های  یادگیری عمیق نیاز به مطالعه عملکرد شبکه‌های عصبی به ویژگی شبکه‌های عصبی گرافی در این زمینه احساس می‌شود. همینطور برخی از پژوهش‌های پیشین نیز از ترکیب روش‌های کلاسیک با شبکه‌های عصبی بهره برده‌اند ولی روشی صرفا مبتنی بر یادگیری عمیق و به صورت یکپارچه\footnote{\lr{End-to-end}} ارائه نشده است.

به طور خلاصه، ما سعی در ارائه یک الگوریتم جهت کشف مجموعه‌های پروتئینی و ماژول‌های عملکردی هم پوشان با دید گرافی به شبکه‌های PPI در کنار استفاده از داده‌های تکمیلی بانک‌های داده‌های زیستی، با روش خوشه‌بندی گراف با گره‌های ویژگی‌دار را داریم. در ادامه اهمیت موضوع، پرسش‌هایی که در این پژوهش به دنبال جواب آن‌ها هستیم و همچنین روش‌ پیشنهادی را به صورت مختصر شرح خواهیم داد.

\section{اهمیت و ضرورت انجام پژوهش}

در حال حاضر زیست شناسان توجه خود را از مطالعه ساختار و عملکرد انفرادی پروتئین‌ها به سمت مطالعه ساختاری و عملکردی ‌‌مجموعه‌های پروتئینی تغییر داده‌اند و مولکول‌های پروتئین‌ها را درون یک شبکه زیستی کلی بررسی می‌کنند \cite{farutin2006edge}.
دلیل این موضوع این است که بررسی یک پروتئین باید در ارتباط با سایر پروتئین‌ها و مجموعه‌های پروتئینی‌ که به آن‌ها تعلق دارد، انجام شود.
از سوی دیگر، جهش‌های موجود در DNA می‌توانند نحوه برهم‌کنش پروتئین‌ها را در یک مجموعه پروتئینی تغییر دهند و به این ترتیب باعث تغییر در عملکرد و رفتار آن مجموعه شوند. این تغییرات نقش مهمی در بررسی فرایند توسعه داروها و همچنین شناخت علل بروز بیماری‌ها دارند \cite{altaf2006development, zaki2018improving}.
به عنوان مثال پژوهش‌های \cite{farutin2006edge, macropol2009rrw}، نشان داده‌اند که برخی از بیماری‌های ژنتیکی به وسیله پروتئین‌های با برهم‌کنش‌های عملکردی مشابه به وجود می‌آیند. همچنین مجموعه‌های پروتئینی به واسطه ارتباط‌‌شان با مسیر‌های زیستی\footnote{Pathway}، برای فهم بهتر نحوه توزیع، جذب، متابولیسم و دفع دارو ضروری هستند. از این روی شناسایی مجموعه‌های پروتئینی برای کشف و توسعه دارو‌ها اهمیت زیادی دارند.
با وجود اهمیت کشف مجموعه‌های پروتئینی، چالش اصلی زمان‌بر و هزینه‌بر بودن فرآیند کشف آن‌ها در آزمایشگاه است که سبب توجه بیشتر محققان به روش‌های محاسباتی جهت کشف ماژول‌های عملکردی شده است \cite{adappi}.

\section{پرسش‌های پژوهش}
% TODO:: COMPLETE THIS SECTION
بعد از تکمیل قسمت روش
\begin{itemize}
    \item  الگوریتم‌های موجود به‌ویژه الگوریتم‌های مبتنی بر GNN‌ برای یافتن انجمن در گراف تا چه حد موفق به کشف شبکه‌های تعامل پروتئین هستند؟
    \item چگونه می‌توان اطلاعات زیستی را به یک الگوریتم موجود برای بهبود کشف ماژول‌های تعاملات پروتئینی اضافه کرد؟
    \item ترکیب اطلاعات توپولوژی و اطلاعات زیستی تا چه حد کشف ماژول‌های پروتئینی را بهبود می‌دهد؟
    \item وجود همپوشانی بین ماژول‌ها در افزایش دقت در کشف ماژول‌های پروتئینی تا چه حد اثرگذار است؟
\end{itemize}
\section{روش پژوهش}
% TODO:: COMPLETE THIS SECTION
\section{جمع‌بندی}
در این فصل به تعریف مسئله پژوهش و اهمیت آن پرداختیم. همچنین به صورت کلی روش پیشنهادی و سوالات پیش‌روی پژوهش را مورد بررسی قرار دادیم.

در ادامه، در فصل دوم به مبانی پایه مورد نیاز جهت فهم بهتر پژوهش اشاره خواهد شد. فصل سوم به معرفی پژوهش‌های پیشین که در زمینه شناسایی ماژول‌های عملکردی و مجموعه پروتئینی برپایه روش‌های محاسباتی هستند اختصاص دارد. فصل چهارم مربوط به روش شناسی و پژوهش است که در ابتدا داده‌های استفاده شده در این پژوهش سپس روش پیشنهادی خود را مطرح می‌کنیم. در فصل پنجم، به بررسی پیاده‌سازی روش پیشنهادی و نتایج حاصل بر اساس معیار‌های ارزیابی می‌پردازیم. همچنین عملکرد روش پیشنهادی را با دیگر روش‌های پیشین مقایسه کرده و برتری روش خود را شرح می‌دهیم. در نهایت، در فصل آخر، پژوهش حاضر را جمع‌بندی می‌کنیم و ایده‌هایی را برای ادامه‌ی مسیر این پژوهش مطرح می‌کنیم.