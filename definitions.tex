\usepackage[top=3.5cm,right=3cm,bottom=4cm,left=3cm]{geometry}  
\usepackage{amsmath, amsthm, amssymb, amscd, latexsym}
\usepackage[marginal,stable,bottom]{footmisc}     % for footnotes: marginal --> the same margins as text, 
                                                                                  %                       stable--> ?
                                                                                  %                       bottom --> starting the footnotes at a fixed place at the            																				%		bottom of the page.
\usepackage{zref-abspage}
\usepackage{perpage}                                             % for footnotes: starting from 1 perpage
\MakePerPage{footnote}
\usepackage{cite}                                                     % for collecting citations: [1,2,3,4] --> [1-4]
\usepackage{setspace}                                            % for switching between double/single space in document
\allowdisplaybreaks                                                    % breaking the lines&pages when needed for the style.
\usepackage{parskip}                                               % ? 

%\relpenalty=9999                                            % show the neccessity of breaks for lines, changing the number to 10000 cause no break in lines.
%\binoppenalty=9999
\usepackage{algorithm, algorithmic}
\usepackage{tabularx, ctable, multirow, bigstrut}
\usepackage{xecolor}
\usepackage{makeidx}
\usepackage{verbatim}
\usepackage[colorlinks,linkcolor=blue,citecolor=blue]{hyperref}
\usepackage{graphicx}
\usepackage{ifthen}
\usepackage{tikz}

\parindent0pt


\makeatletter
\pdfstringdefDisableCommands{%
\let\lr\@firstofone
}
\makeatother

\usepackage{xepersian}

\usepackage[Lenny]{fncychap}

\settextfont[Scale=1.07]{XB Niloofar}
\setlatintextfont[Scale=1.05]{Times New Roman} 
\setdigitfont[Scale=1.05]{Yas} 
% قلم برای اعداد به صورت فارسی با صفر توخالی، در صورتی که بخواهیم اعداد انگلیسی نوشته شوند این خط را غیر فعال می‌کنیم.

\defpersianfont\nastaliq[Scale=2.0]{IranNastaliq}
% قلم برای نوشتن تقدیم
\defpersianfont\nastaliqone[Scale=1.0]{IranNastaliq}
% قلم نستعلیق با سایز متناسب با متن در صورت نیاز
\defpersianfont\anotherfont[Scale=1.2]{XP Ziba}
% قلم برای نوشتن تشکر (قلم فانتزی)
\newenvironment{fantezi}
{\anotherfont }


\makeindex


\def\beginto{
\newpage
\begin{RTL}
\begin{Huge}
\nastaliq

\begin{center}
\vspace*{0.15cm}
تقدیم به 
}

\def\endto{
~
\end{center}
\end{Huge}
\end{RTL}
}

\def\beginthanks{
\newpage

{\centering\Huge{\nastaliqone{
تشکر وقدردانی ~\\
~\\}}}
}

\def\endthanks{
~
}

\def\thanks{
\beginthanks
\input{thanks}
\endthanks
}

\def\startpage{
\newpage
\vspace*{3cm}
\begin{center}
\includegraphics[width=12cm]{besm}
\end{center}
}
\makeatletter
\def\@makechapterhead#1{%
  \vspace*{50\p@}%
  {\parindent \z@ \centering\normalfont
    \ifnum \c@secnumdepth >\m@ne
      \if@mainmatter
        \huge\bfseries \@chapapp\space \tartibi{chapter} 
        \par\nobreak
        \vskip 20\p@
      \fi
    \fi
    \interlinepenalty\@M
    \Huge \bfseries #1\par\nobreak
    \vskip 40\p@
  }}
\def\@makeschapterhead#1{%
  \vspace*{50\p@}%
  {\parindent \z@ \centering
    \normalfont
    \interlinepenalty\@M
    \Huge \bfseries  #1\par\nobreak
    \vskip 40\p@
  }}
\makeatother



\renewcommand\bibname{مراجع}
\def\contentsname{فهرست}

% some extra diffinitions %%%%%%%%%%%%%%%%%
%%%%%%%%%%%%%%%%%%%%%%%%%%%%%

\theoremstyle{plain}
\newtheorem{theorem}{قضیه}[section]
\newtheorem{proposition}[theorem]{گزاره}
\newtheorem{lemma}[theorem]{لم}
\newtheorem{corollary}[theorem]{نتیجه}
\newtheorem{conjecture}[theorem]{حدس}

\theoremstyle{definition}
\newtheorem{definition}[theorem]{تعریف}
\newtheorem{notation}{نماد}
\newtheorem{example}{مثال}
\newtheorem{question}{پرسش}
\newtheorem{problem}{مساله}

\theoremstyle{remark}
\newtheorem{remark}{تبصره}
\newtheorem{point}{تکته}

% -------------------- DON'T EDIT ----------------------------------------
% the following lines are needed for making the appendixes name correct in the index.
\makeatletter
\def\@makechapterhead#1{%
  \vspace*{50\p@}%
  {\parindent \z@ \centering\normalfont
    \ifnum \c@secnumdepth >\m@ne
      \if@mainmatter
        \huge\bfseries \@chapapp\space \thechapter
        \par\nobreak
        \vskip 20\p@
      \fi
    \fi
    \interlinepenalty\@M
    \Huge \bfseries #1\par\nobreak
    \vskip 40\p@  }}
