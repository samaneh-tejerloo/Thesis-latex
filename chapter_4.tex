\chapter{نتایج}

\section{معیار ارزیابی\footnote{\lr{Evaluation metrics/measures}}  موفقیت}
در این قسمت به بررسی معیار‌های ارزیابی عملکرد الگوریتم‌های شناسایی مجموعه‌های 
پروتئینی می‌پردازیم. در بین معیار‌های موجود دو معیار صحت و امتیاز \lr{F} ، بیشترین استفاده را در بین پژوهش‌ها داشته‌اند که ما نیز به منظور تحلیل و مقایسه عملکرد روش خود از آنها استفاده می‌کنیم. برای محاسبه صحت و امتیاز\lr{F} ، در پیش‌بینی مجموعه‌های پروتئینی ابتدا نیاز به آشنایی با معیار شباهت همسایگی داریم:

\subsection{شباهت همسایگی\footnote{\lr{Neighborhood affinity}}}  

با در نظر گرفتن \lr{P} به عنوان مجموعه‌ای از مجموعه‌های پروتئینی شناسایی شده توسط الگوریتم، عملکرد الگوریتم به وسیله تعداد مجموعه‌های پروتئینی مشترک بین \lr{P} و مجموعه‌ای از مجموعه پروتئین‌های مرجع\footnote{\lr{Reference protein complex}}  \lr{B} بدست می‌آید. برای مشخص کردن اینکه آیا یک مجموعه پروتئین شناسایی شده $p\in P$  با یک مجموعه پروتئین مرجع $b\in B$ یکسان هستند یا خیر ما اقدام به محاسبه معیار شباهت همسایگی به صورت مقابل می‌کنیم.
\begin{equation}
	NA(p,b)=\frac{|V_p \cap V_b|^2}{|V_p| \times |V_b|}
\end{equation}

که $V_p$ مجموعه پروتئین‌های حاضر در ترکیب \lr{p} و به طور مشابه $V_b$ مجموعه   پروتئین‌های حاضر در \lr{b} هستند. برای تفسیر شباهت همسایگی یک آستانه\footnote{\lr{Threshold}}  از قبل تعیین شده (معمولا $0.25$) در نظر گرفته می‌شود که شباهت همسایگی‌های بالاتر از آستانه به معنی یکسانی دو مجموعه است. همچنین تعداد مجموعه‌های شناسایی شده‌ای که حداقل با یک مجموعه مرجع یکسان در نظر گرفته می‌شوند را با $N_{cp}$ و تعداد مجموعه های مرجعی که حداقل با یکی از مجموعه‌های شناسایی شده الگوریتمی یکسان در نظر گرفته می‌شوند را با $N_{cb}$ نمایش می‌دهیم \cite{spectral}.
\begin{equation}
	N_{cp}=|\{p|p\in P, \exists b \in B,NA(p,b)\ge \omega\}|
\end{equation} 
\begin{equation}
	N_{cp}=|\{b|b\in B, \exists p \in P,NA(p,b)\ge \omega\}|
\end{equation}

\subsection{امتیاز \lr{F}\footnote{\lr{F-score}}}
امتیاز \lr{F} به کمک دو معیار دیگر یعنی دقت  و بازیابی  محاسبه می‌شود. دقت\footnote{\lr{Precision}} و بازیابی\footnote{\lr{Recall}} را می‌توان به کمک تعاریف بخش قبل به صورت مقابل محاسبه کرد:
\begin{equation}
	\text{Recall} = \frac{N_{cb}}{|B|}
\end{equation}
\begin{equation}
	\text{Precision} = \frac{N_{cp}}{|P|}
\end{equation}

و در نهایت می‌توان امتیاز \lr{F} را به صورت مقابل محاسبه کرد \cite{spectral}  :
\begin{equation}
	F\text{-}score = \frac{2 \times \text{Precision} \times \text{Recall}}{\text{Precision} + \text{Recall}}
\end{equation}

\subsection{صحت\footnote{\lr{Accuracy}}}
معیار صحت به کمک دو معیار دیگر حساسیت خوشه‌بندی\footnote{\lr{Clustering-wise sensitivity (Sn)}}   و ارزش پیش‌بینی مثبت خوشه‌بندی\footnote{\lr{Clustering-wise positive predictive value (PPV)}}  محاسبه می‌شود. با در نظر گرفتن $T_{i,j}$ به عنوان تعداد پروتئین‌هایی که هم در مجموعه پروتئینی مرجع\lr{i} ام و هم در 
مجموعه پروتئینی پیش‌بینی شده \lr{j} ام یافت می‌شوند و همچنین \lr{N} به عنوان تعداد پروتئین‌های مجموعه پروتئینی مرجع \lr{i}، می‌توانیم \lr{Sn} و \lr{PPV} را به صورت مقابل تعریف کنیم:
\begin{equation}
	PPV = \frac{\sum_{j=1}^{|P|} \max_{i=1}^{|B|} \{ T_{ij} \}}{\sum_{j=1}^{|P|} \sum_{i=1}^{|B|} T_{ij}}
\end{equation}
\begin{equation}
	Sn = \frac{\sum_{i=1}^{|B|} \max_{j=1}^{|P|} \{ T_{ij} \}}{\sum_{i=1}^{|B|} N_i}
\end{equation}

در نهایت نیز با کمک این دو عدد می‌توان معیار صحت را به صورت مقابل تعریف کرد :
\begin{equation}
	Acc = \sqrt{Sn \cdot PPV}
\end{equation}

\subsection{تحلیل غنی‌سازی\footnote{\lr{Enrichment analysis}}}
پس از پیدا کردن مجموعه پروتئین‌ها، یک روش ارزیابی عملکرد الگوریتم ارائه شده، بررسی پروتئین‌های موجود در یک مجموعه پروتئینی به منظور تایید عملکرد مشابه آن‌ها به کمک اطلاعات زیستی موجود در مجموعه‌ داده‌هایی مانند \lr{KEGG} (که اطلاعات مسیرهای متابولیکی را در خود ذخیره دارد) و \lr{GO} است. همچنین ابزار‌های آنلاینی مانند \lr{\footnote{\lr{Database for Annotation, Visualization and Integrated Discovery}}DAVID}  ارائه شده‌اند که اطلاعات عملکردی و عملیات تحلیل غنی‌سازی را به صورت خودکار و ساده بر روی مجموعه‌ای از پروتئین‌ها و ژن‌های ورودی انجام می‌دهند. با انجام این تحلیل می‌توانیم از درستی مجموعه‌های پروتئینی پیدا شده اطمینان حاصل کنیم \cite{peerapen2023protein}.
\section{نتایج آزمایش}
همینطور می‌توان از معیار‌های مختص به خوشه‌بندی استفاده نمود تا کیفیت خوشه‌بندی حاصل را مورد ارزیابی قرار داد. به این دسته‌ از معیار‌های ارزیابی معیار‌های درونی\footnote{\lr{Internal evaluation measures}}  می‌گوییم زیرا تنها کیفیت خوشه‌بندی را به کمک خوشه‌های ایجاد شده و ساختار شبکه در نظر می‌گیرند و از داده‌های بیرونی (مانند داده‌های مربوط به مجموعه‌ پروتئین‌های شناخته شده) استفاده نمی‌کنند \cite{clustering}. همینطور 
\
نتایج حاصل از روش پیشنهادی و مقایسه آن با سایر روش‌ها در جدول \ref{table:results} قابل مشاهده می‌باشد. نتایج جدول به خوبی برتری روش پیشنهادی نسبت به سایر روش‌ها در مجموعه داده پروتئین موش‌ از پایگاه داده BioGrid نشان می‌دهد.

