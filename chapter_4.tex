\chapter{روش شناسی پژوهش}

\section{مقدمه}

پژوهش حاضر به شناسایی ماژول‌های عملکردی پروتئینی در شبکه‌های \lr{PPI} با بهره‌گیری از شبکه‌های عصبی گرافی می‌پردازد. همان‌گونه که در بخش بیان مسئله مطرح شد، مسئله شناسایی ماژول‌های عملکردی پروتئینی را می‌توان به‌صورت یک مسئله خوشه‌بندی در شبکه‌های \lr{PPI} مدل‌سازی کرد. از این رو، در فصل پژوهش‌های پیشین، مروری بر مطالعات مرتبط با خوشه‌بندی گراف و روش‌های شناسایی ماژول‌های عملکردی ارائه شد.

بررسی مطالعات پیشین نشان می‌دهد که با وجود پیشرفت‌های قابل توجه در حوزه خوشه‌بندی شبکه‌های زیستی، تعداد پژوهش‌هایی که به‌طور مستقیم از روش‌های مبتنی بر شبکه‌های عصبی گرافی برای تحلیل شبکه‌های \lr{PPI} استفاده کرده‌اند، همچنان محدود است. افزون بر این، تاکنون چارچوب یادگیری یکپارچه و منسجمی که به‌صورت خاص برای شناسایی ماژول‌های عملکردی در شبکه‌های \lr{PPI} طراحی شده باشد، به‌طور کامل مورد توجه قرار نگرفته است. این پژوهش در پی آن است تا با بررسی دقیق‌تر ظرفیت‌ها و قابلیت‌های شبکه‌های عصبی گرافی، گامی در جهت پر کردن این خلأ پژوهشی بردارد.

از آن‌جا که مسئله شناسایی ماژول‌های عملکردی ذاتاً یک مسئله یادگیری بدون نظارت به شمار می‌رود، انتخاب تابع هزینه مناسب و تعیین معیارهای توقف مؤثر نقش بسزایی در تضمین همگرایی و کارایی فرآیند یادگیری ایفا می‌کنند. علاوه بر این، روش پیشنهادی باید قادر باشد به‌صورت هم‌زمان اطلاعات ساختاری شبکه و ویژگی‌های زیستی پروتئین‌ها را به‌طور مؤثر مدل‌سازی کند. در این راستا، در نظر گرفتن همپوشانی میان خوشه‌ها از اهمیت ویژه‌ای برخوردار است؛ چرا که در شبکه‌های زیستی واقعی، ماژول‌های عملکردی پروتئینی اغلب دارای همپوشانی قابل توجهی با یکدیگر هستند. 

این فصل به دو بخش کلی تقسیم می‌شود. در بخش نخست، مجموعه‌داده‌های مورد استفاده معرفی می‌شوند که شامل شبکه‌های \lr{PPI} مورد بررسی، پایگاه داده هستی‌شناسی ژن، مراحل پیش‌پردازش داده‌ها و فرآیند آماده‌سازی آن‌ها است. در بخش دوم، روش پیشنهادی این پژوهش به‌طور جامع ارائه شده و اجزای مختلف آن به‌صورت دقیق مورد تحلیل و بررسی قرار می‌گیرند.

\section{مجموعه داده}
 
 در دهه‌های اخیر، شبکه‌های برهم‌کنش پروتئین–پروتئین (\lr{Protein--Protein Interaction} یا \lr{PPI}) به‌واسطه توسعه روش‌های آزمایشگاهی با توان عملیاتی بالا\footnote{\lr{High-throughput}} به‌طور قابل توجهی گسترش یافته‌اند. از جمله این روش‌ها می‌توان به سیستم‌های دوگانه هیبریدی\footnote{\lr{Two-hybrid systems}} \cite{2015clusteringsurvey} و طیف‌سنجی جرمی\footnote{\lr{Mass spectrometry}} \cite{2018community} اشاره کرد. افزون بر این، روش‌های مبتنی بر متن‌کاوی\footnote{\lr{Text mining}} نیز به‌صورت گسترده برای استخراج تعاملات پروتئینی و ایجاد شبکه‌های \lr{PPI} مورد استفاده قرار گرفته‌اند \cite{gene,2019analysis,2017sparsedegree-corrected}.
 
 از منظر مدل‌سازی گرافی، این منابع داده امکان نمایش شبکه‌های \lr{PPI} را به‌صورت یک گراف بدون جهت $G=(V,E)$ فراهم می‌کنند که در آن هر گره $v \in V$ متناظر با یک پروتئین و هر یال $e \in E$ بیانگر وجود تعامل میان دو پروتئین است. در حالت کلی، منابع داده \lr{PPI} را می‌توان به سه دسته شامل داده‌های آزمایشگاهی، پایگاه‌های داده مبتنی بر روش‌های محاسباتی، و پایگاه‌های داده ادغام‌شده تقسیم‌بندی کرد.  شبکه‌های \lr{PPI} برای گونه‌های زیستی مختلفی نظیر انسان، موش و مخمر در دسترس هستند. با این حال، در این پژوهش تمرکز صرفاً بر شبکه‌های مربوط به گونه مخمر نان \LTRfootnote{Saccharomyces cerevisiae} قرار دارد؛ چرا که بخش عمده‌ای از داده‌های مرجع، مطالعات پیشین و ماژول‌های عملکردی تأییدشده برای این گونه زیستی فراهم شده‌اند. تمامی مجموعه‌داده‌های مورد استفاده مربوط به این گونه بوده و تفاوت آن‌ها عمدتاً در تعداد گره‌ها و یال‌ها و نیز نوع روش استخراج تعاملات است. از جمله پایگاه‌های داده شناخته‌شده مربوط به گونه مخمر‌نان می‌توان به \lr{BioGRID} \cite{2006biogrid}، \lr{DIP} \cite{2002dip} و \lr{Collins} \cite{collins2007} اشاره کرد.
 
 در این پژوهش، شبکه‌های \lr{PPI} نه‌تنها به‌عنوان یک ساختار گرافی، بلکه به‌عنوان ورودی اصلی مدل‌های یادگیری مبتنی بر شبکه‌های عصبی گرافی در نظر گرفته می‌شوند. از این رو، علاوه بر توپولوژی شبکه، وجود وزن یال‌ها و ویژگی‌های معنایی گره‌ها نقش کلیدی در فرآیند یادگیری و استخراج نمایش‌های نهفته ایفا می‌کند. همچنین، به‌منظور ارزیابی عملکرد روش پیشنهادی در شناسایی ماژول‌های عملکردی، از مجموعه‌های مرجع شامل ماژول‌های پروتئینی شناخته‌شده نظیر \lr{CYC2008} و  \cite{2005mips}\lr{MIPS} به‌عنوان معیار صحت‌سنجی استفاده می‌شود.
 
 \subsection{مجموعه‌داده شبکه‌های \lr{PPI}}
 
 شبکه‌های \lr{PPI} مورد استفاده در این پژوهش به‌صورت گراف‌های بدون جهت و وزن‌دار مدل‌سازی می‌شوند که به‌طور رسمی به شکل
 \[
 G = (V, E, W, X)
 \]
 قابل نمایش هستند. در این نمایش، $V$ مجموعه گره‌ها (پروتئین‌ها)، $E$ مجموعه یال‌ها (تعاملات پروتئینی)، $W$ ماتریس وزن یال‌ها و $X$ ماتریس ویژگی‌های گره‌ها است. این چارچوب نمایش، مستقیماً با الزامات ورودی شبکه‌های عصبی گرافی مورد استفاده در روش پیشنهادی سازگار است.
 
 مجموعه‌داده‌های \lr{Collins}، \lr{Gavin}، \lr{Krogan-Core} و \lr{Krogan-Extended} از صفحه مربوط به الگوریتم \lr{IMHRC} \cite{maddi2017discovering} که توسط پژوهشگاه دانش‌های بنیادی (IPM) منتشر شده است، دریافت شده‌اند. این مجموعه‌داده‌ها شامل وزن یال‌ها بوده و بنابراین مستقیماً به‌عنوان گراف‌های وزن‌دار در مدل‌های عصبی گرافی مورد استفاده قرار می‌گیرند.
 
 در مقابل، برای شبکه‌هایی نظیر \lr{BioGRID} و \lr{DIP} که فاقد وزن یال هستند، داده‌ها از پژوهش \lr{AdaPPI} استخراج شده‌اند. از آن‌جا که روش پیشنهادی مبتنی بر یادگیری پیام در شبکه‌های عصبی گرافی نیازمند وزن یال‌ها به‌منظور تنظیم شدت انتشار اطلاعات میان گره‌ها است، وزن هر یال با استفاده از شباهت کسینوسی بین بردارهای ویژگی مبتنی بر هستی‌شناسی ژن (\lr{GO}) مربوط به هر جفت پروتئین محاسبه شده است. این بردارها ماتریس ویژگی گره‌ها $X$ را تشکیل داده و نحوه استخراج آن‌ها در بخش بعدی تشریح خواهد شد.
 
 به‌منظور شفاف‌سازی ساختار داده‌ها، مجموعه‌داده \lr{Collins} به‌عنوان نمونه بررسی می‌شود. هر مجموعه‌داده به‌صورت یک فایل جدولی با فرمت \lr{CSV} ارائه شده و شامل فیلدهای زیر است:
 \begin{itemize}
 	\item \textbf{\lr{protein1}}: شناسه یا نام معنایی\footnote{\lr{Semantic name}} پروتئین اول (گره مبدأ).
 	\item \textbf{protein2}: شناسه یا نام معنایی پروتئین دوم (گره مقصد).
 	\item \textbf{weight}: وزن یال بین دو پروتئین که شدت تعامل یا میزان شباهت زیستی را مدل‌سازی می‌کند.
 \end{itemize}
 

 \subsection{استخراج ویژگی‌ها از پایگاه هستی‌شناسی ژن}
 
 در این بخش، نحوه استخراج ویژگی‌های زیستی مربوط به هر پروتئین با استفاده از پایگاه داده هستی‌شناسی ژن (\lr{Gene Ontology} یا \lr{GO}) تشریح می‌شود. ویژگی‌های استخراج‌شده، ماتریس ویژگی گره‌ها $X$ را در نمایش گرافی شبکه‌های \lr{PPI} تشکیل داده و به‌عنوان ورودی اصلی شبکه عصبی گرافی در روش پیشنهادی مورد استفاده قرار می‌گیرند.
 
 پایگاه داده \lr{GO} شامل سه نسخه اصلی \lr{go-basic}، \lr{go} و \lr{go-plus} است که هر یک از نظر سطح جزئیات و نوع روابط موجود میان عبارات تفاوت دارند. در این پژوهش، نسخه \lr{go-basic} به‌عنوان منبع اصلی استخراج ویژگی‌ها انتخاب شده است؛ چرا که این نسخه شامل روابط سلسله‌مراتبی اصلی بوده و از ایجاد چرخه‌های منطقی جلوگیری می‌کند، که این امر برای استخراج ویژگی‌های پایدار و قابل تفسیر اهمیت دارد. خلاصه‌ای از تفاوت این نسخه‌ها در جدول مربوطه ارائه شده است.
 
 یه جدول طلایی
 
 برای هر گونه‌ی زیستی، یک فایل حاشیه‌نویسی ژن\LTRfootnote{\lr{Gene Annotation File (GAF)}} وجود دارد که نگاشت میان ژن‌ها و عبارات \lr{GO} را مشخص می‌کند. در این پژوهش، تمرکز بر گونه مخمر نان (\lr{Saccharomyces cerevisiae}) است و برای هر ژن، شناسه اختصاصی آن موسوم به \lr{SGD ID} مورد استفاده قرار می‌گیرد. این شناسه امکان استخراج عبارات \lr{GO} متناظر با هر ژن را در سه زیرهستی‌شناسی شامل فرایند زیستی (\lr{Biological Process})، عملکرد مولکولی (\lr{Molecular Function}) و مؤلفه سلولی (\lr{Cellular Component}) فراهم می‌کند.
 
 از آن‌جا که در شبکه‌های \lr{PPI}، پروتئین‌ها اغلب با نام‌های معنایی معرفی می‌شوند، یک گام پیش‌پردازشی برای نگاشت این نام‌ها به \lr{SGD ID} متناظر آن‌ها ضروری است. بدین منظور، از ابزار ارائه‌شده در وب‌سایت \lr{yeastgenome.org} استفاده شده است که امکان تبدیل نام‌های معنایی پروتئین‌های مخمر به شناسه‌های \lr{SGD} را به‌صورت خودکار فراهم می‌کند. پس از انجام این نگاشت، تمامی عبارات \lr{GO} مرتبط با هر پروتئین از فایل حاشیه نویسی ژن استخراج می‌شوند.
 
 در ساده‌ترین حالت، می‌توان هر پروتئین را به‌صورت یک بردار دودویی با بُعدی برابر تعداد کل عبارات \lr{GO} نمایش داد، به‌طوری‌که هر مؤلفه نشان‌دهنده وجود یا عدم وجود یک عبارت \lr{GO} برای آن پروتئین باشد. به‌عنوان نمونه، شبکه \lr{PPI} مربوط به مجموعه‌داده \lr{Collins} شامل 2248 عبارت \lr{GO} متمایز است و در نتیجه، نمایش دودویی ویژگی‌های پروتئین‌ها دارای همین بُعد خواهد بود. با وجود آن‌که نمایش‌های دودویی مبتنی بر \lr{GO} به‌دلیل سادگی و تفسیرپذیری، در بسیاری از پژوهش‌های پیشین مورد استفاده قرار گرفته‌اند، این بردارها به‌طور ذاتی تنک و دارای بُعد بالا هستند که می‌تواند منجر به کاهش کارایی یادگیری در مدل‌های مبتنی بر شبکه‌های عصبی گرافی شود.
 
 بر این اساس، در این پژوهش علاوه بر استفاده از بردارهای ویژگی دودویی، از بردارهای تعبیه‌شده حاصل از روش‌های یادگیری بازنمایی گراف، نظیر \lr{Node2Vec}، نیز بهره گرفته شده است تا اطلاعات معنایی و ساختاری نهفته در روابط میان عبارات \lr{GO} به‌صورت فشرده‌تری در نمایش ویژگی‌ها منعکس شود.
 
 به‌منظور دستیابی به نمایش‌های غنی‌تر و کم‌بعدتر، از رویکرد تعبیه‌سازی عبارات \lr{GO} مشابه روش ارائه‌شده در \cite{ieremie2022transformergo} استفاده شده است. بدین منظور، ابتدا الگوریتم \lr{Node2Vec} بر روی گراف \lr{GO} اعمال شده و برای هر عبارت \lr{GO} یک بردار تعبیه‌ای استخراج می‌شود. سپس، برای هر پروتئین و در هر یک از سه زیرهستی‌شناسی فرایند زیستی، عملکرد مولکولی و مؤلفه سلولی، بردارهای تعبیه‌ای عبارات \lr{GO} متناظر با آن پروتئین میانگین‌گیری می‌شوند. در نتیجه، برای هر زیرهستی‌شناسی یک بردار ویژگی با بُعد 128 به‌دست می‌آید.
 
 
 در نهایت، بردارهای حاصل از سه زیرهستی‌شناسی با یکدیگر الحاق شده و یک بردار ویژگی نهایی با بُعد 384 برای هر پروتئین تشکیل می‌شود. این بردار به‌عنوان نمایش ویژگی گره متناظر با هر پروتئین در ماتریس $X$ قرار گرفته و به‌صورت یکپارچه در فرآیند یادگیری شبکه عصبی گرافی روش پیشنهادی مورد استفاده قرار می‌گیرد.

 
 این جا یه جدول برای ویژگی‌های دیتاست‌ها 
 
 
 \subsection{مجموعه‌های پروتئینی مرجع}
 
\section{روش پیشنهادی}
روش پیشنهادی ما در این پژوهش به کمک استفاده از شبکه‌های عصبی گرافی یک بازنمایی مناسب به منظور خوشه‌بندی شبکه تعاملات پروتئین-پروتئین با در نظر گرفتن ویژگی‌های زیستی و بیان ژنی ایجاد می‌کند که هیمنطور امکان خوشه‌بندی هم‌پوشان را نیز ممکن می‌سازد. روش پیشنهادی از مدل مولد احتمالی برنولی پواسون کمک می‌گیرد و تابع هزینه جدیدی را بر این پایه معرفی می‌کنیم. روش پیشنهادی شامل سه مرحله است که در ادامه به بررسی بیشتر این مراحل می‌پردازیم.

\subsection{مرحله اول: استفاده از شبکه عصبی گرافی به منظور ایجاد ماتریس وابستگی}
با فرض داشتن گراف نود ویژگی‌دار $G$ که می‌توان آن را با دو ماتریس مجاورت $A$  و ویژگی‌ نود‌های $X$ نمایش داد، یک شبکه عصبی گرافی کانولوشنی دو لایه به منظور ایجاد ماتریس وابستگی $F$ در نظر می‌گیریم:
\begin{gather}
	F=GNN_\theta(A,X) \\
	\tilde{A} = A + I_N \\
	\tilde{D}_{ii} = \sum_{j} \tilde{A}_{ij} \\
	\hat{A} = \tilde{D}^{-1/2}\tilde{A}\tilde{D}^{-1/2} \\
	F=ReLU(\hat{A}ReLU(\hat{A}XW^{(1)})W^{(2)})
\end{gather}

مدل شبکه عصبی گرافی پیشنهادی دو تفاوت با شبکه‌های عادی دارد:
\begin{itemize}
	\item استفاده از لایه نرمال‌سازی دسته‌ای بعد از لایه اول گراف کانولوشن
	\item اعمال $L_2$ regularization بر روی ماتریس وزن‌ها ( $W^{(1)}$ و $W^{(2)}$)
\end{itemize}

\subsection{ مرحله دوم: بهینه‌سازی وزن‌های شبکه عصبی گرافی}
در ابتدا باید نگاهی به مفهوم مدل مولد برنولی پواسون داشته باشیم، این مدل سعی بر بازسازی گراف به کمک ماتریس وابستگی $F$ به صورت مقابل دارد:
\begin{equation}
	A_{uv} \sim Bernoulli(1-e^{-F_uF_v^T})
\end{equation}
حال می‌توان با استفاده از مدل برنولی پواسون به محاسبه $p(A|F)$ یا likelihood با فرمولاسیون مقابل عمل کنیم:
\begin{equation}
	P(A|F) = \prod_{A_{uv}\in E} 1-e^{-F_uF_v^T} \times \prod_{A_{uv}\notin E} e^{-F_uF_v^T}
\end{equation}

در مرحله بعد  به منظور ایجاد تابع هزینه، اقدام به اعمال $-log$ می‌کنیم.
در نتیجه به فرمول مقابل می‌رسیم:
\begin{equation}
	-logp(A|F)=-\sum_{A_{uv}\in E}
	​log(1-exp(-F_uF_v^T))+\sum_{A_{uv}\notin E}
	​F_uF_v^T
\end{equation}
حال می‌توانیم ادعا کنیم که تابع هزینه ما با $-log(A|F)$ برابر می‌کند.
\begin{equation}
	L(F)=-\sum_{A_{uv}\in E}
	​log(1-exp(-F_uF_v^T))+\sum_{A_{uv}\in E}
	​F_uF_v^T
\end{equation}
تابع هزینه این است که ماتریس همسایگی در بیشتر موارد یک ماتریس به شدت تنک می‌باشد. از این روی مقدار عبارت دوم در رابطه بیشتر از قسمت اول می‌شود. به همین دلیل اقدام به استفاده از مقدار امید ریاضی هر یک از عبارات با توزیع یکنواخت بر روی تمامی یال‌ها می‌کنیم.
\begin{equation}
	L(F)=-E_{(U,V)\sim P_E}[log(1-exp(-F_uF_v^T))]+E_{(u,v)\sim P_N}[F_uF_v^T]
\end{equation}
که در آن $P_E$ توزیع یکنواخت بر روی یال‌ها و $P_N$ یک توزیع یکنواخت بر روی دو راسی است که بین آن‌ها یال وجود ندارد. در نهایت می‌توان تابع هزینه حاصل را به صورت مقابل نمایش داد:
\begin{equation}
	\theta^*=argmin_{\theta}\: \:\:L(GNN_θ(A,X)) + \lambda_1||W^{(1)}||_2 +\lambda_2 ||W^{(2)}||_2
\end{equation}
\subsection{ مرحله سوم: تخصیص نود‌ها به خوشه‌ها}
در نهایت با پیدا کردن پارامتر‌های مدل، اقدام به پیش‌بینی ماتریس وابستگی $F$ می‌کنیم و برای تخصیص نود‌ها به خوشه‌ها یک آستانه $\varphi$ در نظر می گیریم:
\begin{equation}
	F_{uc} = 
	\begin{cases}
		1 & \text{if}\: F_{uc}>\varphi \\
		0 & \text{otherwise} 
	\end{cases}
\end{equation}

\section{جمع‌بندی}