\chapter{روش}
در این فصل به بررسی روش پیشنهادی بر پایه شبکه‌های عصبی گرافی به منظور خوشه‌بندی شبکه تعاملات پروتئین-پروتئین می‌پردازیم. در ابتدای این فصل نگاهی به مجموعه داده‌های موجود می‌کنیم و دلیل انتخاب آن‌ها را برای آزمایش روش پیشنهادی بررسی می‌کنیم.‌

\section{مجموعه داده}
در دهه گذشته، داده‌های \lr{PPI} از طریق روش‌های آزمایشگاهی با خروجی بالا\footnote{\lr{High-throughput}}  مانند سیستم‌های دوگانه هیبریدی\footnote{\lr{Two-hybrid systems}}  \cite{2015clusteringsurvey}، طیف‌سنجی جرمی\footnote{\lr{Mass spectrometry}}  \cite{2018community} به‌شدت غنی‌ شده‌اند. همچنین، روش‌های متن کاوی\footnote{\lr{Text mining}}  برای ایجاد شبکه‌های \lr{PPI} نیز به صورت گسترده استفاده شده‌اند \cite{gene} \cite{2019analysis}\cite{2017sparsedegree-corrected}. به طور کلی می‌توان منابع داده \lr{PPI} را به دسته‌های آزمایشگاهی، پایگاه داده‌های ایجاد شده به کمک روش‌های محاسباتی و همچنین پایگاه‌ داده‌های ادغام شده تقسیم بندی کرد. به عنوان مثال می‌توان به برخی از این مجموعه داده‌های تعامل پروتئین پروتئین مانند  \lr{Biogrid} \cite{2006biogrid}, \lr{DIP} \cite{2002dip}, \lr{Collins} \cite{collins2007} و \lr{MIPS} \cite{2005mips} اشاره کرد. برای صحت سنجی از مجموعه‌های پروتئینی یافت شده نیز از مجموعه داده‌هایی شامل مجموعه‌های پروتئینی شناخته شده مانند \lr{CYC2008} و یا \lr{MIPS} می‌توان استفاده کرد.  

\section{روش پیشنهادی}
روش پیشنهادی ما در این پژوهش به کمک استفاده از شبکه‌های عصبی گرافی یک بازنمایی مناسب به منظور خوشه‌بندی شبکه تعاملات پروتئین-پروتئین با در نظر گرفتن ویژگی‌های زیستی و بیان ژنی ایجاد می‌کند که هیمنطور امکان خوشه‌بندی هم‌پوشان را نیز ممکن می‌سازد. روش پیشنهادی از مدل مولد احتمالی برنولی پواسون کمک می‌گیرد و تابع هزینه جدیدی را بر این پایه معرفی می‌کنیم. روش پیشنهادی شامل سه مرحله است که در ادامه به بررسی بیشتر این مراحل می‌پردازیم.

\subsection{مرحله اول: استفاده از شبکه عصبی گرافی به منظور ایجاد ماتریس وابستگی}
با فرض داشتن گراف نود ویژگی‌دار $G$ که می‌توان آن را با دو ماتریس مجاورت $A$  و ویژگی‌ نود‌های $X$ نمایش داد، یک شبکه عصبی گرافی کانولوشنی دو لایه به منظور ایجاد ماتریس وابستگی $F$ در نظر می‌گیریم:
\begin{gather}
	F=GNN_\theta(A,X) \\
	\tilde{A} = A + I_N \\
	\tilde{D}_{ii} = \sum_{j} \tilde{A}_{ij} \\
	\hat{A} = \tilde{D}^{-1/2}\tilde{A}\tilde{D}^{-1/2} \\
	F=ReLU(\hat{A}ReLU(\hat{A}XW^{(1)})W^{(2)})
\end{gather}

مدل شبکه عصبی گرافی پیشنهادی دو تفاوت با شبکه‌های عادی دارد:
\begin{itemize}
	\item استفاده از لایه نرمال‌سازی دسته‌ای بعد از لایه اول گراف کانولوشن
	\item اعمال $L_2$ regularization بر روی ماتریس وزن‌ها ( $W^{(1)}$ و $W^{(2)}$)
\end{itemize}

\subsection{ مرحله دوم: بهینه‌سازی وزن‌های شبکه عصبی گرافی}
در ابتدا باید نگاهی به مفهوم مدل مولد برنولی پواسون داشته باشیم، این مدل سعی بر بازسازی گراف به کمک ماتریس وابستگی $F$ به صورت مقابل دارد:
\begin{equation}
	A_{uv} \sim Bernoulli(1-e^{-F_uF_v^T})
\end{equation}
حال می‌توان با استفاده از مدل برنولی پواسون به محاسبه $p(A|F)$ یا likelihood با فرمولاسیون مقابل عمل کنیم:
\begin{equation}
	P(A|F) = \prod_{A_{uv}\in E} 1-e^{-F_uF_v^T} \times \prod_{A_{uv}\notin E} e^{-F_uF_v^T}
\end{equation}

در مرحله بعد  به منظور ایجاد تابع هزینه، اقدام به اعمال $-log$ می‌کنیم.
در نتیجه به فرمول مقابل می‌رسیم:
\begin{equation}
	-logp(A|F)=-\sum_{A_{uv}\in E}
	​log(1-exp(-F_uF_v^T))+\sum_{A_{uv}\notin E}
	​F_uF_v^T
\end{equation}
حال می‌توانیم ادعا کنیم که تابع هزینه ما با $-log(A|F)$ برابر می‌کند.
\begin{equation}
	L(F)=-\sum_{A_{uv}\in E}
	​log(1-exp(-F_uF_v^T))+\sum_{A_{uv}\in E}
	​F_uF_v^T
\end{equation}
تابع هزینه این است که ماتریس همسایگی در بیشتر موارد یک ماتریس به شدت تنک می‌باشد. از این روی مقدار عبارت دوم در رابطه بیشتر از قسمت اول می‌شود. به همین دلیل اقدام به استفاده از مقدار امید ریاضی هر یک از عبارات با توزیع یکنواخت بر روی تمامی یال‌ها می‌کنیم.
\begin{equation}
	L(F)=-E_{(U,V)\sim P_E}[log(1-exp(-F_uF_v^T))]+E_{(u,v)\sim P_N}[F_uF_v^T]
\end{equation}
که در آن $P_E$ توزیع یکنواخت بر روی یال‌ها و $P_N$ یک توزیع یکنواخت بر روی دو راسی است که بین آن‌ها یال وجود ندارد. در نهایت می‌توان تابع هزینه حاصل را به صورت مقابل نمایش داد:
\begin{equation}
	\theta^*=argmin_{\theta}\: \:\:L(GNN_θ(A,X)) + \lambda_1||W^{(1)}||_2 +\lambda_2 ||W^{(2)}||_2
\end{equation}
\subsection{ مرحله سوم: تخصیص نود‌ها به خوشه‌ها}
در نهایت با پیدا کردن پارامتر‌های مدل، اقدام به پیش‌بینی ماتریس وابستگی $F$ می‌کنیم و برای تخصیص نود‌ها به خوشه‌ها یک آستانه $\varphi$ در نظر می گیریم:
\begin{equation}
	F_{uc} = 
	\begin{cases}
		1 & \text{if}\: F_{uc}>\varphi \\
		0 & \text{otherwise} 
	\end{cases}
\end{equation}