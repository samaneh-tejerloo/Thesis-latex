\chapter{روش شناسی پژوهش}

\section{مقدمه}

پژوهش حاضر به شناسایی ماژول‌های عملکردی پروتئینی در شبکه‌های \lr{PPI} با بهره‌گیری از شبکه‌های عصبی گرافی می‌پردازد. همان‌گونه که در بخش بیان مسئله مطرح شد، مسئله شناسایی ماژول‌های عملکردی پروتئینی را می‌توان به‌صورت یک مسئله خوشه‌بندی در شبکه‌های \lr{PPI} مدل‌سازی کرد. از این رو، در فصل پژوهش‌های پیشین، مروری بر مطالعات مرتبط با خوشه‌بندی گراف و روش‌های شناسایی ماژول‌های عملکردی ارائه شد.

بررسی مطالعات پیشین نشان می‌دهد که با وجود پیشرفت‌های قابل توجه در حوزه خوشه‌بندی شبکه‌های زیستی، تعداد پژوهش‌هایی که به‌طور مستقیم از روش‌های مبتنی بر شبکه‌های عصبی گرافی برای تحلیل شبکه‌های \lr{PPI} استفاده کرده‌اند، همچنان محدود است. افزون بر این، تاکنون چارچوب یادگیری یکپارچه و منسجمی که به‌صورت خاص برای شناسایی ماژول‌های عملکردی در شبکه‌های \lr{PPI} طراحی شده باشد، به‌طور کامل مورد توجه قرار نگرفته است. این پژوهش در پی آن است تا با بررسی دقیق‌تر ظرفیت‌ها و قابلیت‌های شبکه‌های عصبی گرافی، گامی در جهت پر کردن این خلأ پژوهشی بردارد.

از آن‌جا که مسئله شناسایی ماژول‌های عملکردی ذاتاً یک مسئله یادگیری بدون نظارت به شمار می‌رود، انتخاب تابع هزینه مناسب و تعیین معیارهای توقف مؤثر نقش بسزایی در تضمین همگرایی و کارایی فرآیند یادگیری ایفا می‌کنند. علاوه بر این، روش پیشنهادی باید قادر باشد به‌صورت هم‌زمان اطلاعات ساختاری شبکه و ویژگی‌های زیستی پروتئین‌ها را به‌طور مؤثر مدل‌سازی کند. در این راستا، در نظر گرفتن همپوشانی میان خوشه‌ها از اهمیت ویژه‌ای برخوردار است؛ چرا که در شبکه‌های زیستی واقعی، ماژول‌های عملکردی پروتئینی اغلب دارای همپوشانی قابل توجهی با یکدیگر هستند. 

این فصل به دو بخش کلی تقسیم می‌شود. در بخش نخست، مجموعه‌داده‌های مورد استفاده معرفی می‌شوند که شامل شبکه‌های \lr{PPI} مورد بررسی، پایگاه داده هستی‌شناسی ژن، مراحل پیش‌پردازش داده‌ها و فرآیند آماده‌سازی آن‌ها است. در بخش دوم، روش پیشنهادی این پژوهش به‌طور جامع ارائه شده و اجزای مختلف آن به‌صورت دقیق مورد تحلیل و بررسی قرار می‌گیرند.

\section{مجموعه داده}
 
 در دهه‌های اخیر، شبکه‌های برهم‌کنش پروتئین-پروتئین به‌واسطه توسعه روش‌های آزمایشگاهی با توان عملیاتی بالا\LTRfootnote{\lr{High-throughput}} به‌طور قابل توجهی گسترش یافته‌اند. از جمله این روش‌ها می‌توان به سیستم‌های دوگانه هیبریدی\LTRfootnote{\lr{Two-hybrid systems}} \cite{2015clusteringsurvey} و طیف‌سنجی جرمی\LTRfootnote{\lr{Mass spectrometry}} \cite{2018community} اشاره کرد. افزون بر این، روش‌های مبتنی بر متن‌کاوی\LTRfootnote{\lr{Text mining}} نیز به‌صورت گسترده برای استخراج تعاملات پروتئینی و ایجاد شبکه‌های \lr{PPI} مورد استفاده قرار گرفته‌اند \cite{gene,2019analysis,2017sparsedegree-corrected}.
 
 از منظر مدل‌سازی گرافی، این منابع داده امکان نمایش شبکه‌های \lr{PPI} را به‌صورت یک گراف بدون جهت $G=(V,E)$ فراهم می‌کنند که در آن هر گره $v \in V$ متناظر با یک پروتئین و هر یال $e \in E$ بیانگر وجود تعامل میان دو پروتئین است. در حالت کلی، منابع داده \lr{PPI} را می‌توان به سه دسته شامل داده‌های آزمایشگاهی، پایگاه‌های داده مبتنی بر روش‌های محاسباتی، و پایگاه‌های داده ادغام ‌شده تقسیم‌بندی کرد.  شبکه‌های \lr{PPI} برای گونه‌های زیستی مختلفی نظیر انسان، موش و مخمر در دسترس هستند. با این حال، در این پژوهش تمرکز صرفاً بر شبکه‌های مربوط به گونه مخمر نان\LTRfootnote{Saccharomyces cerevisiae} قرار دارد؛ چرا که بخش عمده‌ای از داده‌های مرجع، مطالعات پیشین و ماژول‌های عملکردی تأیید شده برای این گونه زیستی فراهم شده‌اند. تمامی مجموعه ‌داده‌های مورد استفاده مربوط به این گونه بوده و تفاوت آن‌ها عمدتاً در تعداد گره‌ها و یال‌ها و نیز نوع روش استخراج تعاملات است. از جمله پایگاه‌های داده شناخته‌ شده مربوط به گونه مخمر‌نان می‌توان به \lr{BioGRID} \cite{2006biogrid}، \lr{DIP} \cite{2002dip} و \lr{Collins} \cite{collins2007} اشاره کرد.
 
 در این پژوهش، شبکه‌های \lr{PPI} نه ‌تنها به‌عنوان یک ساختار گرافی، بلکه به‌عنوان ورودی اصلی مدل‌های یادگیری مبتنی بر شبکه‌های عصبی گرافی در نظر گرفته می‌شوند. از این رو، علاوه بر توپولوژی شبکه، وجود وزن یال‌ها و ویژگی‌های معنایی گره‌ها نقش کلیدی در فرآیند یادگیری و استخراج نمایش‌های نهفته ایفا می‌کند. همچنین، به‌منظور ارزیابی عملکرد روش پیشنهادی در شناسایی ماژول‌های عملکردی، از مجموعه‌های مرجع شامل ماژول‌های پروتئینی شناخته‌شده نظیر \lr{CYC2008}\cite{cyc2008} و  \lr{MIPS}\cite{2005mips} به‌عنوان معیار صحت‌سنجی استفاده می‌شود.
 
 \subsection{مجموعه ‌داده شبکه‌های \lr{PPI}}
 
 شبکه‌های \lr{PPI} مورد استفاده در این پژوهش به‌صورت گراف‌های بدون جهت و وزن‌دار مدل‌سازی می‌شوند که به‌طور رسمی به شکل
 $
 G = (V, E, W, X)
 $
 قابل نمایش هستند. در این نمایش، $V$ مجموعه گره‌ها (پروتئین‌ها)، $E$ مجموعه یال‌ها (تعاملات پروتئینی)، $W$ ماتریس وزن یال‌ها و $X$ ماتریس ویژگی‌های گره‌ها است. این چارچوب نمایش، مستقیماً با الزامات ورودی شبکه‌های عصبی گرافی مورد استفاده در روش پیشنهادی سازگار است.
 
 مجموعه‌ داده‌های \lr{Collins}، \lr{Gavin}، \lr{Krogan-Core} و \lr{Krogan-Extended} از مقاله مربوط به الگوریتم \lr{IMHRC} \cite{maddi2017discovering} که توسط پژوهشگاه دانش‌های بنیادی (IPM) منتشر شده است، دریافت شده‌اند. این مجموعه‌ داده‌ها شامل وزن یال‌ها بوده و بنابراین مستقیماً به‌عنوان گراف‌های وزن‌دار در مدل‌های عصبی گرافی مورد استفاده قرار می‌گیرند.
 
 در مقابل، برای شبکه‌هایی نظیر \lr{BioGRID} و \lr{DIP} که فاقد وزن یال هستند، داده‌های مورد استفاده از پژوهش \lr{AdaPPI} استخراج شده‌اند. اگرچه مجموعه‌های مربوطه از طریق وب‌سایت اصلی این پایگاه‌ها نیز قابل دسترسی هستند، اما به‌منظور انجام مقایسه‌ای منصفانه با روش \lr{AdaPPI}، از همان داده‌های ارائه‌شده در این پژوهش استفاده شده است.
 از آن‌جا که روش پیشنهادی مبتنی بر یادگیری پیام در شبکه‌های عصبی گرافی نیازمند وزن یال‌ها به‌منظور تنظیم شدت انتشار اطلاعات میان گره‌ها است، وزن هر یال با استفاده از شباهت کسینوسی بین بردارهای ویژگی مبتنی بر عبارات هستی‌شناسی ژن مربوط به هر جفت پروتئین محاسبه شده است. این بردارها ماتریس ویژگی گره‌ها $X$ را تشکیل داده و نحوه استخراج آن‌ها در بخش بعدی تشریح خواهد شد.
 
 به‌ منظور شفاف‌سازی ساختار داده‌ها، مجموعه ‌داده \lr{Collins} به‌عنوان نمونه بررسی می‌شود. هر مجموعه ‌داده به‌صورت یک فایل جدولی با فرمت \lr{CSV} ارائه شده و شامل فیلدهای زیر است:
 \begin{itemize}
 	\item \textbf{\lr{protein1}}: شناسه یا نام معنایی\LTRfootnote{\lr{Semantic name}} پروتئین اول (گره مبدأ).
 	\item \textbf{\lr{protein2}}: شناسه یا نام معنایی پروتئین دوم (گره مقصد).
 	\item \textbf{weight}: وزن یال بین دو پروتئین که شدت تعامل یا میزان شباهت زیستی را مدل‌سازی می‌کند.
 \end{itemize}
 

 \subsection{استخراج ویژگی‌ها از پایگاه هستی‌شناسی ژن}
 
 در این بخش، نحوه‌ی استخراج ویژگی‌های زیستی مربوط به هر پروتئین با استفاده از پایگاه داده هستی‌شناسی ژن تشریح می‌شود. ویژگی‌های استخراج‌ شده، ماتریس ویژگی گره‌ها $X$ را در نمایش گرافی شبکه‌های \lr{PPI} تشکیل داده و به‌عنوان ورودی اصلی شبکه عصبی گرافی در روش پیشنهادی مورد استفاده قرار می‌گیرند.
 
 پایگاه داده \lr{Gene Ontology (GO)} در سه نسخه اصلی شامل \lr{go-basic}، \lr{go} و \lr{go-plus} ارائه می‌شود که هر یک از نظر میزان جزئیات و همچنین نوع روابط برقرار میان عبارات، تفاوت‌هایی دارند. در این پژوهش، نسخه \lr{go-basic} به‌عنوان منبع اصلی برای استخراج ویژگی‌ها انتخاب شده است. دلیل این انتخاب آن است که این نسخه عمدتاً شامل روابط سلسله‌مراتبی پایه بوده و ساختار ساده‌تر و منسجم‌تری نسبت به سایر نسخه‌ها دارد. 
 از سوی دیگر، در نسخه‌های جامع‌تر نظیر \lr{go} و \lr{go-plus}، انواع مختلفی از یال‌ها با تفاوت‌های معنایی قابل توجه وجود دارد؛ به‌گونه‌ای که اجرای الگوریتم \lr{Node2Vec} بر روی چنین گرافی نمی‌تواند به‌صورت معنادار و قابل تفسیر انجام شود. بنابراین، استفاده از نسخه \lr{go-basic} امکان استخراج تعبیه‌های پایدارتر و قابل تفسیرتر را فراهم می‌کند. 
 خلاصه‌ای از تفاوت میان این نسخه‌ها در جدول \ref{tab:go-versions} ارائه شده است.
 
 برای هر گونه‌ی زیستی، یک فایل حاشیه‌نویسی ژن\LTRfootnote{\lr{Gene Annotation File (GAF)}} وجود دارد که نگاشت میان ژن‌ها و عبارات \lr{GO} را مشخص می‌کند. در این پژوهش، تمرکز بر گونه مخمر نان است و برای هر ژن، شناسه اختصاصی آن موسوم به \lr{SGD ID} مورد استفاده قرار می‌گیرد. این شناسه امکان استخراج عبارات \lr{GO} متناظر با هر ژن را در سه زیرهستی‌شناسی شامل فرآیند زیستی، عملکرد مولکولی و مؤلفه سلولی فراهم می‌کند.
   \renewcommand{\arraystretch}{1.6}
 \begin{table}
 	\caption{تفاوت نسخه‌های هستی شناسی ژن}
 	\label{tab:go-versions}
 	\centering
 	\begin{tabular}{c|m{10cm}}
 		\textbf{نوع} & 
 		\begin{center}
 			\textbf{توضیحات}
 		\end{center}
 		\\
 		\hline
 		\footnotesize\textbf{go-basic} & 
 		\footnotesize{
 			نسخهٔ فیلترشدهٔ \lr{GO} که به‌صورت قطعی بدون دور است. در این نسخه، به دلیل عدم وجود دور، برچسب‌ها به‌راحتی به گره‌های والد نسبت داده می‌شوند. انواع یال‌ها در این نسخه شامل \lr{part\_of}، \lr{regulates}، \lr{positively regulates} و \lr{negatively regulates} هستند. همچنین، در این نسخه سه زیرگراف \lr{MF}، \lr{BP} و \lr{CC} به‌طور کامل از یکدیگر جدا بوده و هیچ یالی میان گره‌های این زیرگراف‌ها وجود ندارد.}
 		\\ \hline
 		\footnotesize\textbf{go} &
 		\footnotesize{
 			هستهٔ اصلی \lr{GO} است که نسبت به نسخهٔ پایه، دو نوع یال \lr{has\_part} و \lr{occurs\_in} را نیز شامل می‌شود. وجود این روابط باعث ایجاد دور در گراف شده و میان زیرگراف‌ها ارتباط برقرار می‌کند.
 		}
 		\\ \hline
 		\footnotesize\textbf{go-plus} &
 		\footnotesize{
 			نسخهٔ کامل‌تری از \lr{GO} است که ارتباط با سایر هستی‌شناسی‌ها، از جمله \lr{ChEBI}، \lr{Cell Ontology} و \lr{Uberon} را نیز شامل می‌شود و دارای مجموعهٔ کاملی از روابط است که صرفاً به \lr{GO} محدود نمی‌شود.
 		}
 	\end{tabular}
 	\vspace{5mm}
 \end{table}

\vspace{-10mm}
 از آن‌جا که در شبکه‌های \lr{PPI}، پروتئین‌ها اغلب با نام‌های معنایی معرفی می‌شوند، یک گام پیش‌پردازشی برای نگاشت این نام‌ها به \lr{SGD ID} متناظر آن‌ها ضروری است. بدین منظور، از ابزار ارائه‌شده در وب‌سایت \lr{yeastgenome.org} استفاده شده است که امکان تبدیل نام‌های معنایی پروتئین‌های مخمر به شناسه‌های \lr{SGD} را به‌صورت خودکار فراهم می‌کند. پس از انجام این نگاشت، تمامی عبارات \lr{GO} مرتبط با هر پروتئین از فایل حاشیه نویسی ژن استخراج می‌شوند.
 
 در ساده‌ترین حالت، می‌توان هر پروتئین را به‌صورت یک بردار دودویی با بعدی برابر تعداد کل عبارات \lr{GO} نمایش داد، به‌طوری‌که هر مؤلفه نشان‌دهنده وجود یا عدم وجود یک عبارت \lr{GO} برای آن پروتئین باشد. به‌عنوان نمونه، شبکه \lr{PPI} مربوط به مجموعه‌ داده \lr{Collins} شامل 2248 عبارت \lr{GO} متمایز است و در نتیجه، نمایش دودویی ویژگی‌های پروتئین‌ها دارای همین بعد خواهد بود. با وجود آن‌که نمایش‌های دودویی مبتنی بر \lr{GO} به‌دلیل سادگی و تفسیرپذیری، در بسیاری از پژوهش‌های پیشین مورد استفاده قرار گرفته‌اند، این بردارها به‌طور ذاتی تنک و دارای بعد بالا هستند که می‌تواند منجر به کاهش کارایی یادگیری در مدل‌های مبتنی بر شبکه‌های عصبی گرافی شود.

 بر این اساس، در این پژوهش علاوه بر استفاده از بردارهای ویژگی دودویی، از بردارهای تعبیه‌شده حاصل از روش‌های یادگیری تعبیه گراف، نظیر \lr{Node2Vec}، نیز بهره گرفته شده است تا اطلاعات معنایی و ساختاری نهفته در روابط میان عبارات \lr{GO} به‌صورت فشرده‌تری در نمایش ویژگی‌ها منعکس شود.
 
 به‌منظور دستیابی به نمایش‌های غنی‌تر و کم‌بعدتر، از رویکرد تعبیه‌سازی عبارات \lr{GO} مشابه روش ارائه‌شده در \cite{ieremie2022transformergo} استفاده شده است. بدین منظور، ابتدا الگوریتم \lr{Node2Vec} بر روی گراف \lr{GO} اعمال شده و برای هر عبارت \lr{GO} یک بردار تعبیه‌ای استخراج می‌شود. در پژوهش حاضر، الگوریتم \lr{Node2Vec} به‌طور مستقیم بر روی گراف \lr{GO} اجرا نشده است. در عوض، از تعبیه‌های از پیش محاسبه‌شده‌ای استفاده شده است که از مخزن گیت‌هاب\LTRfootnote{https://github.com/marcopodda/go-node2vec} استخراج گردیده‌اند. در این مخزن، الگوریتم \lr{Node2Vec} بر روی گراف \lr{go-basic} اعمال شده و برای هر یک از اصطلاحات \lr{GO} در زیرفضاهای \lr{BP}، \lr{MF} و \lr{CC}، یک بردار تعبیه مستقل با بُعد ۱۲۸ یادگیری شده است. 
 
 ابتدا، عبارات تعبیه \lr{GO} هر پروتئین در هر زیرفضا میانگین‌گیری شده و در نهایت، بردارهای حاصل از سه زیرهستی‌شناسی با یکدیگر الحاق شده و یک بردار ویژگی نهایی با بعد 384 برای هر پروتئین تشکیل می‌شود. این بردار به‌عنوان نمایش ویژگی گره متناظر با هر پروتئین در ماتریس $X$ قرار گرفته و به‌صورت یکپارچه در فرآیند یادگیری شبکه عصبی گرافی روش پیشنهادی مورد استفاده قرار می‌گیرد.

 
 این جا یه جدول برای ویژگی‌های دیتاست‌ها 
 
 
 \subsection{مجموعه‌های پروتئینی مرجع و پروتکل ارزیابی}
 به‌منظور ارزیابی و صحت‌سنجی عملکرد الگوریتم پیشنهادی در شناسایی ماژول‌های عملکردی پروتئینی، استفاده از مجموعه‌های پروتئینی مرجع ضروری است تا بتوان نتایج به‌ دست‌ آمده را با دانش زیستی موجود مقایسه کرد. مجموعه‌های پروتئینی مرجع وابسته به گونه زیستی بوده و برای گونه مخمر نان  چندین مجموعه مرجع شناخته‌ شده در دسترس است که از جمله مهم‌ترین آن‌ها می‌توان به \lr{MIPS} و \lr{CYC2008} اشاره کرد.
 در مقایسه روش‌های محاسباتی مختلف، انتخاب مجموعه پروتئینی مرجع مورد استفاده نقش تعیین‌کننده‌ای در تفسیر نتایج ایفا می‌کند. از این رو، با توجه به این‌که روش پایه مورد استفاده برای مقایسه با روش پیشنهادی، چارچوب \lr{AdaPPI} است، در این پژوهش نیز از همان مجموعه‌های پروتئینی مرجع به‌کاررفته در آن مطالعه استفاده شده است. این مجموعه‌ ترکیبی از سایر مجموعه‌های پروتئینی شامل \lr{MIPS}، \lr{CYC2008}، \lr{Aloy}، \lr{SGD} و \lr{TAP06} می‌باشد.
 
 این مجموعه پروتئینی مرجع به‌صورت یک فایل متنی ذخیره شده است که در آن، هر خط متناظر با یک ماژول پروتئینی بوده و شامل نام‌های معنایی پروتئین‌های عضو آن ماژول است که با  خط فاصله از یکدیگر جدا شده‌اند. شکل \ref{fig:gold-standard} نمونه‌ای از ساختار این مجموعه مرجع را نشان می‌دهد.
 
 تصویر
 
 به‌منظور ارزیابی مجموعه‌های پروتئینی استخراج ‌شده از هر شبکه \lr{PPI}، ابتدا تنها پروتئین‌هایی که در شبکه \lr{PPI} متناظر حضور دارند در هر مجموعه مرجع حفظ می‌شوند. سپس، مجموعه‌هایی که تعداد پروتئین‌های آن‌ها کمتر از سه عضو باشد حذف می‌گردند. این فرآیند پیش‌پردازش، مطابق با پروتکل ارزیابی به‌کاررفته در روش \lr{AdaPPI} انجام شده و امکان مقایسه منصفانه و سازگار نتایج روش پیشنهادی با روش‌های مرجع را فراهم می‌کند.
 
 \section{روش پیشنهادی}
 در بخش پیشین، مجموعه‌داده‌های مورد استفاده در این پژوهش و نیز فرآیند استخراج ویژگی‌ها به‌صورت جامع مورد بررسی قرار گرفت. در این بخش، ابتدا چارچوب کلی روش پیشنهادی برای شناسایی ماژول‌های عملکردی پروتئینی معرفی می‌شود و سپس پیاده‌سازی‌ها و تغییرات اعمال‌شده بر این چارچوب مورد تحلیل قرار می‌گیرند.
 بدین منظور، ابتدا چارچوب کلی روش پیشنهادی و نحوهٔ توسعهٔ آن بر پایهٔ مدل \lr{NOCD}  که به‌طور مفصل‌ در فصل ‌پژوهش‌های پیشین بررسی شده است، تشریح می‌شود. در ادامه، معماری‌های مختلف شبکه‌های عصبی گرافی به‌کاررفته در این پژوهش معرفی و تحلیل خواهند شد. در گام بعد، تأثیر انواع ویژگی‌های ورودی بر عملکرد مدل مورد بررسی قرار می‌گیرد و در نهایت، روش ترکیبی پیشنهادی ارائه و تشریح می‌شود.
 
 \subsection{چارچوب کلی روش پیشنهادی}
 
 در چارچوب پیشنهادی، همان‌گونه که در شکل \ref{fig:overall-architecture} مشاهده می‌شود، ابتدا گراف تعامل پروتئین-پروتئین به همراه ویژگی‌های متناظر با هر پروتئین به‌عنوان ورودی به یک شبکه عصبی گرافی داده می‌شود. خروجی این شبکه، یک ماتریس وابستگی به‌صورت 
 $F \in \mathbb{R}^{n \times c}$ 
 است که در آن $n$ نشان‌دهنده تعداد پروتئین‌ها و $c$ بیانگر تعداد خوشه‌ها می‌باشد. هر سطر از این ماتریس میزان تعلق یک پروتئین به خوشه‌های مختلف را نشان می‌دهد.
 
  \begin{figure}[h]
 	\centering
 	\includegraphics[width=0.7\textwidth]{figures/overall-architecture.png}
 	\caption{نمایی از چارچوب کلی روش پیشنهادی}
 	\label{fig:overall-architecture}
 \end{figure}
انتخاب ویژگی‌های ورودی نیز نقش مهمی در عملکرد مدل ایفا می‌کند. در این پژوهش، انواع مختلف ویژگی‌های ورودی مورد ارزیابی قرار گرفته‌اند که شامل ویژگی‌های باینری مبتنی بر اصطلاحات \lr{GO}، تعبیه برداری اصطلاحات \lr{GO} و همچنین استفاده همزمان از هر دو نوع ویژگی می‌باشد. نتایج حاصل از این بررسی‌ها در بخش‌های بعدی ارائه و تحلیل خواهند شد.

 همچنین، به‌ منظور آموزش شبکه عصبی گرافی، تعریف یک تابع هزینه مناسب امری ضروری است. در روش پیشنهادی، تابع هزینه پایه مشابه روش \lr{NOCD} در نظر گرفته شده است، با این تفاوت که در این پژوهش نسخه‌ای وزن‌دار از آن معرفی شده است. اعمال وزن در تابع هزینه موجب بهبود عملکرد مدل در شناسایی ساختارهای همپوشان در گراف تعامل پروتئین-پروتئین شده است که جزئیات آن در ادامه این فصل تشریح خواهد شد. علاوه بر این، انتخاب معماری مناسب برای شبکه عصبی گرافی یکی از چالش‌های اصلی این پژوهش بوده است. در این تحقیق، تأثیر معماری‌های مختلف شبکه‌های عصبی گرافی بر عملکرد شناسایی مجموعه‌های پروتئینی مورد بررسی و مقایسه قرار گرفته است.
  
در انتها، روش \lr{NOCD} برای استخراج خوشه‌ها نیاز به تعیین یک مقدار آستانه مشخص بر روی ماتریس وابستگی دارد. با این حال، انتخاب آستانه مناسب همواره چالش‌برانگیز بوده و می‌تواند بر نتایج نهایی تأثیر قابل توجهی داشته باشد. از این رو، در روش پیشنهادی این محدودیت حذف شده و خوشه‌ها بر اساس مجموعه‌ای از آستانه‌های مختلف استخراج می‌شوند. این رویکرد امکان تحلیل پایداری خوشه‌ها و کاهش وابستگی نتایج به یک مقدار آستانه خاص را فراهم می‌کند که دلایل و مزایای آن در ادامه مورد بحث قرار می‌گیرد.

\subsection{تابع هزینه}

در این بخش، تابع هزینه پیشنهادی این پژوهش تشریح می‌شود. به‌عنوان نقطه شروع، از تابع هزینه روش \lr{NOCD} استفاده می‌کنیم که سازوکار آن به‌طور مفصل در فصل پژوهش‌های پیشین مورد بررسی قرار گرفته است. این تابع هزینه مبتنی بر یک مدل مولد احتمالی بوده و هدف آن یادگیری ماتریس وابستگی هم‌پوشان گره‌ها به ماژول‌ها است.

تابع هزینه \lr{NOCD} به‌صورت زیر تعریف می‌شود:
\begin{equation}
	L(F) = -\sum_{(u,v)\in E}
	\log\left(1-\exp(-F_u F_v^{\top})\right)
	+ \sum_{(u,v)\notin E}
	F_u F_v^{\top},
	\label{eq:nocd-loss-unbalanced}
\end{equation}
که در آن $F \in \mathbb{R}^{|V|\times K}$ ماتریس وابستگی گره‌ها به $K$ ماژول پنهان است، $F_u$ بردار وابستگی گره $u$، و $E$ مجموعه یال‌های موجود در گراف می‌باشد. جمله اول تابع هزینه متناظر با جفت ‌گره‌هایی است که بین آن‌ها یال وجود دارد و احتمال مشاهده تعامل را بیشینه می‌کند، در حالی که جمله دوم به جفت‌ گره‌های بدون یال مربوط بوده و به‌عنوان یک منظم‌کننده برای جلوگیری از ایجاد تعاملات کاذب عمل می‌کند.

به‌منظور بهبود عملکرد این تابع هزینه در زمینه شناسایی ماژول‌های عملکردی پروتئینی در شبکه‌های \lr{PPI}، در این پژوهش دو توسعه متفاوت بر روی تابع هزینه \lr{NOCD} مورد بررسی قرار گرفته است. توسعه نخست مبتنی بر بازسازی ماتریس همسایگی مرتبه دوم به‌جای ماتریس همسایگی مرتبه اول است و توسعه دوم به وارد کردن وزن تعاملات پروتئین-پروتئین در فرمول‌بندی تابع هزینه اختصاص دارد. در ادامه، هر یک از این توسعه‌ها به‌تفصیل تشریح می‌شوند.

\subsubsection{بازسازی ماتریس همسایگی مرتبه دوم}

ایده استفاده از ماتریس همسایگی مرتبه دوم از تحلیل ساختار شبکه‌های \lr{PPI} و بررسی مجموعه‌های پروتئینی مرجع الهام گرفته شده است. بررسی تجربی این شبکه‌ها نشان می‌دهد که پروتئین‌های عضو یک ماژول عملکردی لزوماً دارای تعامل مستقیم (یال مرتبه اول) با یکدیگر نیستند، بلکه در بسیاری از موارد، ارتباط آن‌ها از طریق مسیرهای کوتاه با طول دو یا حتی سه یال برقرار می‌شود. به‌ویژه، وجود تعاملات مرتبه دوم میان پروتئین‌های یک ماژول به‌طور گسترده مشاهده می‌شود.

بر این اساس، به‌جای استفاده صرف از ماتریس همسایگی مرتبه اول $A$، از ماتریس همسایگی مرتبه دوم برای تعریف ساختار تعاملات استفاده می‌شود. ماتریس همسایگی مرتبه دوم به‌صورت $A^2 = AA$ محاسبه شده و سپس به‌صورت یک ماتریس دودویی آستانه‌گذاری می‌شود:
\begin{equation}
	\hat{A}_{ij} =
	\begin{cases}
		1 & \text{if } \; \; A^2_{ij} > 0, \\
		0 & \text{otherwise}
	\end{cases}
	\label{eq:nocd-A2}
\end{equation}

در این حالت، وجود یک مقدار غیرصفر در $A^2_{ij}$ نشان‌دهنده وجود حداقل یک مسیر با طول دو میان گره‌های $i$ و $j$ است. در نسخه اصلاح‌ شده تابع هزینه، به‌جای استفاده از ماتریس همسایگی $A$ برای تعیین مجموعه یال‌ها $E$، از ماتریس $\hat{A}$ استفاده می‌شود. این تغییر امکان مدل‌سازی تعاملات غیرمستقیم اما زیستیِ معنادار را در فرآیند یادگیری فراهم می‌کند.

علاوه بر این، به‌منظور ایجاد تعادل میان اطلاعات تعاملات مستقیم و غیرمستقیم، یک نسخه ترکیبی از تابع هزینه نیز در نظر گرفته شده است که به‌صورت یک ترکیب محدب از دو تابع هزینه مبتنی بر $A$ و $A^2$ تعریف می‌شود:
\begin{equation}
	L(F) = (1-\lambda)\, L_{A}(F) + \lambda\, L_{A^2}(F),
	\label{eq:nocd-A2-convex}
\end{equation}
که در آن $\lambda \in (0,1)$ یک ابرپارامتر تنظیم‌کننده است و میزان تأثیر تعاملات مرتبه دوم را در فرآیند یادگیری کنترل می‌کند. این فرمول‌بندی انعطاف‌پذیری لازم را برای سازگاری تابع هزینه با ساختارهای متفاوت شبکه‌های \lr{PPI} فراهم می‌آورد.

\subsubsection{استفاده از وزن تعاملات}
یکی دیگر از توسعه‌های اعمال‌ شده در این پژوهش، وزن‌دار کردن تعاملات پروتئین-پروتئین در فرآیند یادگیری ماتریس وابستگی $F$ است. انگیزه اصلی این رویکرد، در نظر گرفتن میزان اطمینان و شدت تعاملات زیستی و کاهش اثر یال‌های ضعیف یا نویزی در فرآیند شناسایی ماژول‌های عملکردی می‌باشد. 

بدین منظور، همان‌گونه که در بخش مجموعه‌ داده‌ها توضیح داده شد، برای شبکه‌های \lr{PPI} وزن‌دار از وزن‌های ارائه ‌شده در خود مجموعه‌ داده استفاده شده است. در مقابل، برای شبکه‌هایی که فاقد وزن تعاملات هستند، وزن هر تعامل به کمک شباهت کسینوسی میان بردارهای تعبیه‌ی ویژگی‌های \lr{GO} مربوط به دو پروتئین محاسبه شده است. این رویکرد امکان وارد کردن دانش عملکردی زیستی مستقل از ساختار گراف را به تابع هزینه فراهم می‌سازد و به‌ویژه در شبکه‌های بدون وزن، نقش مؤثری در هدایت فرآیند یادگیری ایفا می‌کند.

به‌منظور لحاظ کردن وزن تعاملات در چارچوب \lr{NOCD}، تابع هزینه این روش به صورت وزن‌دار و مطابق با معادله \ref{eq:nocd-weighted} بازنویسی شده است. در این فرمول‌بندی، وزن $w_{uv}$ بیانگر قدرت یا میزان اطمینان تعامل بین دو پروتئین $u$ و $v$ می‌باشد. بدیهی است در صورتی که بین دو پروتئین $u,v \in V$ هیچ‌گونه تعامل ثبت‌شده‌ای وجود نداشته باشد، مقدار
$
w_{uv} = 0
$
در نظر گرفته می‌شود.
\vspace{-4mm}
\begin{equation}
	L(F) = -\sum_{u,v \in V} w_{uv} \log\left(1-\exp\left(-F_u F_v^T\right)\right)
	+ \sum_{u,v \in V} (1-w_{uv}) F_u F_v^T
	\label{eq:nocd-weighted}
\end{equation}
\vspace{-4mm}
در این تابع هزینه، تعاملاتی با وزن بالاتر سهم بیشتری در جمله نخست داشته و مدل را به بازسازی دقیق‌تر این یال‌ها ترغیب می‌کنند، در حالی که تعاملات با وزن پایین یا صفر، تأثیر کمتری در فرآیند یادگیری داشته و عملاً مشابه یال‌های غایب در نظر گرفته می‌شوند. برخلاف تابع هزینه \lr{NOCD} که تمامی تعاملات مشاهده‌ شده را به‌صورت هم‌ارزش مدل‌سازی می‌کند، نسخه وزن‌دار پیشنهادی امکان تمایز بین تعاملات با درجات مختلف اعتبار زیستی را فراهم کرده و به بهبود کیفیت شناسایی ماژول‌های عملکردی در شبکه‌های \lr{PPI} منجر می‌شود.

در پایان باید ذکر کرد که در فصل بعد یعنی فصل نتایج، عملکرد تجربی هر یک از نسخه‌های معرفی شده از تابع هزینه \lr{NOCD} به صورت کمی مورد بررسی و مقایسه قرار می‌گیرند.

\subsection{شبکه‌های عصبی گرافی}
انتخاب معماری مناسب شبکه عصبی گرافی نقش تعیین‌کننده‌ای در عملکرد الگوریتم شناسایی ماژول‌های عملکردی ایفا می‌کند، چرا که نوع مدل گرافی به‌طور مستقیم بر نحوه استخراج و انتشار اطلاعات ساختاری و ویژگی‌محور در شبکه اثرگذار است. از همین رو، در این پژوهش به بررسی و مقایسه چندین معماری متداول و پیشرفته از شبکه‌های عصبی گرافی پرداخته شده است. معماری‌های مورد مطالعه شامل شبکه‌های عصبی گرافی مبتنی بر کانولوشن، مدل‌های توجه‌محور و شبکه‌های دارای سازوکار پرش دانش می‌باشند. در ادامه، نقش هر یک از این معماری‌ها در بهبود فرآیند شناسایی ماژول‌های عملکردی به‌صورت مجزا در شبکه‌های \lr{PPI} مورد بررسی قرار می‌گیرد.



\subsubsection{شبکه عصبی گرافی پیچشی}
نحوه‌ی عملکرد شبکه‌های عصبی گرافی پیچشی در فصل مفاهیم بنیادی و به‌ویژه در بخش مفاهیم محاسباتی به‌صورت تفصیلی مورد بررسی قرار گرفته است. در این بخش، تمرکز اصلی بر معرفی و تحلیل ابرپارامترهایی است که برای معماری شبکه‌ی پیشنهادی در نظر گرفته شده‌اند.
یکی از مهم‌ترین ابرپارامترهای این معماری، تعداد لایه‌های شبکه است؛ زیرا این پارامتر به‌صورت مستقیم با پدیده‌ی هموارسازی بیش‌از‌حد\LTRfootnote{Over-smoothing} یا هموارسازی ناکافی\LTRfootnote{Under-smoothing} در ارتباط است. علاوه بر آن، ابعاد بازنمایی ویژگی‌ها نیز یکی از دیگر ابرپارامترهای کلیدی محسوب می‌شود که نقش مهمی در کیفیت استخراج ویژگی‌ها از گراف ایفا می‌کند.

در مجموعه‌ای از آزمایش‌های تکمیلی، تأثیر استفاده از لایه‌ی نرمال‌سازی دسته‌ای\LTRfootnote{Batch Normalization Layer}، عبارت منظم‌سازی \lr{L2}\LTRfootnote{L2 Regularization Term} در تابع هزینه، استفاده از اتصالات باقیمانده\LTRfootnote{Residual Connections} و نیز دراپ‌اوت\LTRfootnote{Dropout} در ساختار شبکه مورد آزمایش قرار گرفت. نتایج این ارزیابی‌ها نشان داد که افزودن هر یک از این اجزا به معماری پیشنهادی، منجر به کاهش نسبی عملکرد مدل شده است.
در فصل نتایج و تحلیل، تأثیر کیفی و کمی هر یک از این ابرپارامترها بر عملکرد نهایی روش پیشنهادی به‌صورت دقیق‌تر مورد بحث و مقایسه قرار خواهد گرفت.

\subsubsection{شبکه‌های عصبی گرافی توجه‌محور}
در شبکه‌های عصبی گرافی توجه‌محور، افزون بر ابرپارامترهایی چون تعداد لایه‌ها و ابعاد بازنمایی، تعداد سرهای سازوکار توجه نیز نقش بسیار مهمی در عملکرد مدل ایفا می‌کند. بر این اساس، در مجموعه آزمایش‌های انجام‌ شده، تأثیر تعداد سرهای متفاوت شامل ۱، ۴ و ۸ سر توجه به‌طور جداگانه نیز بررسی شده است.

 به‌طور کلی، انتظار می‌رود که عملکرد شبکه‌های عصبی گرافی توجه‌محور در مقایسه با شبکه‌های عصبی گرافی پیچشی بهبود قابل‌توجهی داشته باشد. دلیل این امر آن است که در شبکه‌های عصبی گرافی پیچشی، در فرآیند به‌روزرسانی بازنمایی یک پروتئین، تمامی همسایگان با وزن یکسان در نظر گرفته می‌شوند. در مقابل، شبکه‌های عصبی گرافی توجه‌محور با یادگیری ضرایب توجه، میزان اهمیت نسبی هر یک از همسایگان را به‌صورت پویا مدل‌سازی می‌کنند و در نتیجه، بازنمایی غنی‌تر و دقیق‌تری بر اساس میزان تأثیر هر همسایه ایجاد می‌نمایند.
 
 علاوه بر نسخه‌ی پایه‌ی شبکه‌ی عصبی گرافی توجه‌محور، در این بخش معماری‌های پیشرفته‌تری از این خانواده، از جمله \lr{SuperGAT} و \lr{GATv2}، نیز مورد بررسی قرار گرفته‌اند تا تأثیر این ساختارهای بهبودیافته بر عملکرد مدل به‌صورت دقیق و نظام‌مند ارزیابی شود.
 

 
 \subsubsection{شبکه عصبی گرافی دارای سازوکار پرش دانش}
 با توجه به این واقعیت که در شبکه‌های \lr{PPI}، پروتئین‌های مختلف برای دستیابی به بازنمایی‌های مؤثر، به اطلاعات همسایگی در مرتبه‌های متفاوتی نیاز دارند، بررسی عملکرد سازوکار \lr{JKNet} در این پژوهش ضروری به نظر می‌رسد. این سازوکار امکان تجمیع بازنمایی‌های حاصل از لایه‌های مختلف شبکه عصبی گرافی را فراهم می‌کند و بدین ترتیب، وابستگی هر گره به عمق‌های متفاوت شبکه به‌طور تطبیقی مدل‌سازی می‌شود. در این راستا، نسخه‌های گوناگونی از \lr{JKNet} شامل \lr{Max Pooling}، \lr{Concatenation} و \lr{Bi-LSTM Attention} مورد بررسی و ارزیابی قرار گرفته‌اند. در ادامه، مروری کلی بر هر یک از این روش‌ها ارائه می‌شود:
 \begin{itemize}
 	\item \textbf{بیشینه‌گیری:}  
 	در این روش، بردارهای بازنمایی حاصل از هر لایه شبکه عصبی برای هر پروتئین، به‌صورت مؤلفه‌به‌مؤلفه با یکدیگر بیشینه‌گیری می‌شوند و بردار حاصل به‌عنوان بازنمایی نهایی آن پروتئین در نظر گرفته می‌شود.
 	
 	\item \textbf{الحاق:}  
 	در این رویکرد، بازنمایی‌های تولید شده در لایه‌های مختلف شبکه عصبی گرافی با یکدیگر الحاق شده و بردار به‌دست‌آمده به‌عنوان بازنمایی نهایی پروتئین مورد استفاده قرار می‌گیرد.
 
 \item \textbf{سازوکار توجه مبتنی بر شبکه عصبی \lr{Bi-LSTM}:}  
 در این روش، برای هر پروتئین، بازنمایی‌های حاصل از لایه‌های مختلف شبکه به‌صورت یک توالی به یک شبکه عصبی بازگشتی دوسویه از نوع \lr{Bi-LSTM} داده می‌شوند. سپس با بهره‌گیری از سازوکار توجه، میزان اهمیت هر بازنمایی لایه‌ای به‌صورت تطبیقی یاد گرفته شده و در نهایت، با ترکیب وزن‌دار این بازنمایی‌ها، بازنمایی نهایی هر پروتئین تولید می‌شود.
 \end{itemize}
 شایان ذکر است که سازوکار \lr{JKNet} مستقل از معماری شبکه عصبی گرافی پایه بوده و قابلیت ترکیب با مدل‌های مختلف را دارد. در این پژوهش، شبکه عصبی گرافی پایه از نوع \lr{GAT} در نظر گرفته شده است.

\subsection{توابع فعال‌سازی}
توابع فعال‌سازی نقش اساسی در شبکه‌های عصبی ایفا می‌کنند، زیرا با ایجاد غیرخطی‌بودن در مدل، امکان یادگیری الگوها و روابط پیچیده میان داده‌ها را فراهم می‌سازند. ازاین‌رو، در این پژوهش تأثیر انتخاب تابع فعال‌سازی بر عملکرد روش پیشنهادی مورد بررسی و تحلیل قرار گرفته است. توابع فعال‌سازی ارزیابی‌شده در این مطالعه شامل موارد زیر هستند:
\begin{itemize}
	\item \textbf{ReLU\LTRfootnote{Rectified Linear Unit}:}  
	این تابع با صفر کردن مقادیر منفی و عبور مستقیم مقادیر مثبت، به یادگیری پایدار و کاهش مشکل ناپدیدشدن گرادیان کمک می‌کند:
	\begin{equation}
		\mathrm{ReLU}(x) = \max(0, x)
	\end{equation}
	
	\item \textbf{GELU\LTRfootnote{Gaussian Error Linear Unit}:}  
	تابع \lr{GELU} با وزن‌دهی نرم به ورودی‌ها بر اساس توزیع نرمال، رفتار غیرخطی پیوسته‌تری نسبت به \lr{ReLU} ایجاد می‌کند:
	\begin{equation}
		\begin{aligned}
			& \mathrm{GELU}(x) = x \, \Phi(x) 
			= x \cdot \frac{1}{2} \left(1 + \mathrm{erf}\!\left(\frac{x}{\sqrt{2}}\right)\right) \\
			& erf(x) = \frac{2}{\sqrt{\pi}}\int_0^x e^{-t^2}dt
		\end{aligned}
	\end{equation}
	
	\item \textbf{ELU\LTRfootnote{Exponential Linear Unit}:}  
	این تابع با اختصاص مقادیر منفی اشباع‌شده برای ورودی‌های منفی، میانگین خروجی‌ها را به صفر نزدیک کرده و همگرایی شبکه را بهبود می‌بخشد:
	\begin{equation}
		\mathrm{ELU}(x) =
		\begin{cases}
			x, & x > 0, \\
			\alpha \left(e^{x} - 1\right), & x \le 0
		\end{cases}
	\end{equation}
\end{itemize}

شایان ذکر است که با توجه به ماهیت تابع هزینه پایه مورد استفاده در این پژوهش، یعنی \lr{NOCD}، که نیازمند ماتریس وابستگی $F$ با شرط نامنفی بودن عناصر آن به‌صورت
$\forall i,j;\;F_{ij} \ge 0$
می‌باشد، تابع فعال‌سازی لایه نهایی شبکه در تمامی آزمایش‌ها به‌صورت ثابت از نوع \lr{ReLU} در نظر گرفته شده است.


\subsection{استخراج مجموعه‌های پروتئینی و تعیین آستانه}
در روش پیشنهادی این پژوهش، به‌منظور استخراج ماژول‌های عملکردی از ماتریس وابستگی به‌دست‌آمده از شبکه آموزش‌دیده، از تعیین یک مقدار آستانه ثابت استفاده نشده است. در عوض، مجموعه‌های متناظر با آستانه‌های مختلف در بازه
$0.2, 0.3, \ldots, 0.7$
به‌صورت مستقل استخراج می‌شوند. سپس اجتماع این مجموعه‌های پروتئینی محاسبه شده و تنها مجموعه‌های یکتا نگه داشته می‌شوند تا از ایجاد مجموعه‌های تکراری جلوگیری شود.  

انگیزه اصلی این رویکرد همانطور که در تصویر \ref{} قابل مشاهده است، وجود هم‌پوشانی میان مجموعه‌های پروتئینی است. به‌گونه‌ای که با بررسی ماژول‌های موجود در مجموعه مرجع مشاهده می‌شود برخی ماژول‌های کوچک‌تر به‌طور کامل درون چند ماژول بزرگ‌تر قرار گرفته‌اند. بنابراین، انتخاب تنها یک آستانه مشخص قادر به نمایش مناسب این ساختار هم‌پوشان نبوده و استفاده از چندین آستانه امکان شناسایی دقیق‌تر این روابط تو در تو را فراهم می‌کند.

یه تصویر

\subsection{ویژگی‌های ورودی}

 
\section{جمع‌بندی}