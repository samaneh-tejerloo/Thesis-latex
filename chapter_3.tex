\chapter{بررسی منابع}
در این قسمت به بررسی پژوهش‌های پیشینی که به منظور پیدا کردن مجموعه‌های پروتئینی در شبکه‌های \lr{PPI} انجام شده‌اند، می‌پردازیم. همانطور که در بخش‌های پیشین بررسی شد، تمرکز این پژوهش بر روی دید گرافی به شبکه‌های \lr{PPI} و ادغام اطلاعات زیست‌شناسی پروتئین‌ها به منظور تشخیص دقیق‌تر مجموعه‌های پروتئینی است. از آنجایی که پیدا کردن مجموعه‌های پروتئینی در شبکه‌های \lr{PPI} معادل خوشه‌بندی این شبکه‌ها می‌باشد، ما ابتدا چند نمونه از پژوهش‌های مرتبط با خوشه‌بندی گراف‌های دارای گره‌ ویژگی که بیشترین ارتباط را با هدف پژوهش ما دارند را بررسی می‌کنیم. 
\section{خوشه‌بندی گراف‌های دارای گره ویژگی}
پژوهش وحید جان‌نثاری و همکارانش\cite{nonnegative} ، یک الگوریتم بر پایه تجزیه نامنفی ماتریسی\footnote{\lr{Non-negative matrix factorization}}  به منظور خوشه‌بندی گراف‌های ویژگی‌دار معرفی می‌کند. روش آن‌ها ابتدا اطلاعات ساختاری که توسط ماتریس همسایگی\footnote{\lr{Adjacency matrix}}  نشان داده می‌شود را به کمک تجزیه نامنفی متقارن ماتریس\footnote{\lr{Symmetric non-negative matrix factorization}}  و اطلاعات ویژگی‌های گره‌ها را به کمک تجزیه نامنفی بازتابی ماتریس\footnote{\lr{Projective non-negative matrix factorization}}  به یک فضای کم بعد مختص خوشه‌بندی (هم بعد با تعداد خوشه‌ها) به صورت جداگانه انتقال می‌دهد که درجه عضویت هر گره به هر خوشه را نمایش می‌دهد. همینطور به منظور حفظ ثبات در خوشه بندی در هر دو فضا اقدام به نزدیک کردن این دو ماتریس به کمک تابع هدف می‌کند که به صورت مقابل تعریف شده است:
\begin{gather}
    J_{of}=min||A-VV^{T}||_{F}^{2}+\alpha||VV^{T}-UU^{T}||_{F}^{2}+||F-UU^{T}F||_{F}^2 \\
    s.t.\:\: V\ge 0,\: U\ge 0,\:V^TV=I,\: U^TU=I.\nonumber  
\end{gather}

که در تابع هدف، \lr{A} ماتریس همسایگی، $V\in R^{n\times k}$ ماتریس حاصل از تجزیه نامنفی متقارن ماتریس \lr{A} است. همینطور با در نظر  گرفتن \lr{M} (ماتریس شباهت\footnote{\lr{Similarity matrix}}  گره‌ها براساس ماتریس ویژگی‌ها) به صورت $M=UU^T;U\in R^{n\times k}$ و عبارت سوم در بهینه سازی که به صورت مقابل بیان شده است:  $|F-MF|$ در واقع اقدام به استفاده از ویژگی خودبیانگری\footnote{\lr{Self-expression }}  داده‌ها کرده‌اند، که در نتیجه روش بیان شده را می‌توان یک روش ترکیبی از خوشه‌بندی زیر فضا\footnote{\lr{Subspace clustering}}  و تجزیه نامنفی ماتریس در نظر گرفت.

در پژوهشی دیگر توسط کانگ و همکارانش\cite{fine} ، یک روش بر پایه شبکه‌های پیچشی گرافی\footnote{\lr{Graph convolutional networks}}  و خوشه‌بندی طیفی ارائه شده است. ایده اصلی در این روش بر پایه پردازش سیگنالی گراف است که در آن یک فیلتر پایین گذر\footnote{\lr{Low-pass filter}}  را به منظور نزدیک کردن و ادغام ویژگی‌های گره‌ها و ساختار گراف به ماتریس ویژگی‌ها اعمال می‌کنند. در نتیجه یک بازنمایی جدید بر این اساس را برای گره‌ها بدست می‌آورند:
\begin{equation}
\bar{X}=(I-1/2 L)^k X    
\end{equation}

همچنین در فرمول بالا \lr{k} یک هایپر پارامتر است که میزان مرتبه مجاورت بازنمایی به دست آمده را مشخص می‌کند به عبارت دیگر مقادیر کوچک‌تر \lr{k} دید محلی تری به ساختار گراف دارند و بالعکس. $L=I-A$ که \lr{L}  ماتریس لاپلاسی نرمال شده\footnote{\lr{Normalized Laplacian matrix}}  می‌باشد. در مرحله بعد برای اعمال خوشه‌بندی طیفی، نیاز به محاسبه ماتریس شباهت بین گره‌ها است که به صورت مقابل عمل کرده‌اند.

\begin{equation}
\min_S||\bar{X} ^T- \bar{X} ^T S||_F^2+α||S-f(A)||_F^2    
\end{equation}


که در اینجا ماتریس شباهت \lr{S} از بهینه سازی تابع هدف بالا بدست می‌آید و سپس با یک انتقال به یک ماتریس متقارن نامنفی تبدیل شده و در نهایت نیز خوشه‌بندی طیفی روی آن اعمال می‌شود. یکی از مشکلات این روش انتخاب مناسب هایپر پارامتر \lr{K} است که به طور مستقیم بر خروجی الگوریتم تاثیر می‌گذارد که توسط پژوهش دیگری که توسط ژانگ و همکارانش\cite{adaptive}  انجام شده‌ است، دو استراتژی \lr{AGC} و \lr{IAGC} برای پیدا کردن مقدار مناسب \lr{k} ارائه شده است.

یکی از مشکلات روش‌های بر پایه بازنمایی این است که دو فرآیند بازنمایی‌ها داده‌ها و خوشه‌بندی از یکدیگر مستقل‌اند در نتیجه نمی‌توان اطمینان داشت که بازنمایی‌های ایجاد شده برای وظیفه مورد‌نظر (در اینجا خوشه‌بندی) مناسب هستند و همچنین نمی‌توان الگوریتم بازنمایی را بر اساس خطای خوشه‌بندی به طور مناسب به روزرسانی نمود. از این روی، وانگ و همکاران\cite{2019attributed}  یک روش خوشه بندی یکپارچه توجه محور بر پایه شبکه عصبی گراف ارائه داده‌اند که مرحله بازنمایی و خوشه‌بندی را با هم ترکیب می‌کند. در این پژوهش از یک شبکه گرافی توجه محور\footnote{\lr{Graph attentional}}  به عنوان کدگذار  استفاده شده‌ است. ضرایب توجه کدگذار\footnote{\lr{Decoder}} با استفاده از یک ماتریس مجاورت با مرتبه بالا همانند پژوهش قبلی محاسبه می‌شوند. قسمت کدگشا\footnote{\lr{Encoder}}  نیز از ضرب داخلی بردار‌های بازنمایی کدگذار به منظور بازسازی ماتریس مجاورت گراف استفاده می‌کند که برای خروجی این قسمت تابع هزینه بازسازی در نظر گرفته شده است. نوآوری این مقاله در معرفی مفهوم بازنمایی خود بهینه‌ساز است که در آن به طور مکرر نقاط مربوط به هر خوشه براساس مقدار اطمینان تعلق به خوشه به‌روزرسانی می‌شود و به طور همزمان بازنمایی‌ها را نیز به وسیله آن اصلاح می‌کند.

\section{پیش‌بینی مجموعه‌های پروتئینی}

در ادامه به بررسی روش‌های استفاده شده به منظور پیش‌بینی مجموعه‌های پروتئینی در شبکه‌های \lr{PPI} می‌پردازیم و یک دسته‌بندی برای این روش‌ها ارائه می‌دهیم. به طور کلی الگوریتم‌های پیش‌بینی مجموعه‌های پروتئینی را می‌توان به دو دسته تقسیم کرد:

\textbf{روش‌های بر پایه شبکه \footnote{\lr{Network-based methods}}: }\\
این روش‌ها تنها بر ساختار شبکه \lr{PPI} تمرکز می‌کنند. که به دو زیر دسته تقسیم می‌شوند:
\begin{itemize}
    \item روش‌های تقسیمی \footnote{\lr{Divisive methods }}: این دسته از روش‌ها، شبکه را به زیر شبکه‌ها تقسیم می‌کنند و این عمل را تا رسیدن به درجه دلخواه خوشه بندی تکرار می‌کنند. معروف ترین الگوریتم‌ این دسته الگوریتم خوشه‌بندی مارکوف\footnote{\lr{Markov clustering algorithm}}  \cite{2012markov} است که زیر شبکه‌ها را به کمک قدم تصادفی\footnote{\lr{Random walk}}  در شبکه پیدا می‌کند.
    \item روش‌های تجمعی \footnote{\lr{Agglomerative methods}}: با مجموعه کوچکی از پروتئین‌ها شروع کرده و با ترکیب آن‌ها اقدام به پیدا کردن مجموعه‌های پروتئینی نهایی می‌کند. الگوریتم \lr{CPNM} \cite{2020motif} یکی از الگوریتم‌‌های این دسته است که از امبدینگ موتیف‌های\footnote{\lr{Motif}}  شبکه به منظور پیدا کردن نقش پروتئین‌ها استفاده می‌کند. سپس به منظور ایجاد بردار ویژگی‌ پروتئین‌ها از آن‌ها استفاده می‌شود. در نهایت نیز از روش پیدا کردن همسایگان به منظور شناسایی مجموعه‌های پروتئینی استفاده می‌کند. یکی دیگر از الگوریتم‌های تجمعی معروف الگوریتم \lr{ClusterONE} \cite{2012clusterone} است. این الگوریتم ابتدا پروتئین‌های با درجه‌ بالاتر را به عنوان پروتین‌های هسته \footnote{\lr{Seed}} (پروتین‌های آغازین)‌ در نظر گرفته می‌گیرد. سپس زیرگروه‌هایی از گره‌ها با بیشترین انسجام برای گره‌های هسته انتخاب می‌شوند. در انتها نیز گره هسته از بین گره‌هایی که مربوط به یک ترکیب شناخته شده نیستند انتخاب می‌شوند و این مراحل تکرار می‌شوند تا همه پروتئین‌ها به یک ترکیب مرتبط شوند. الگوریتم دیگر، \lr{MCODE}\footnote{\lr Molecular complex detection}  \cite{2003mcode} است که در سه مرحله انجام می‌شود. این الگوریتم ابتدا گره‌ها را وزن دهی می‌کند، سپس به شناسایی مجموعه‌ها می‌پردازد و در انتها نیز اقدام به اضافه / حذف کردن پروتئين‌ها به/از مجموعه‌های شناسایی شده با توجه به یک معیار اتصال می‌کند.
\end{itemize}

\textbf{روش‌های مبتنی بر آگاهی از زمینه‌های زیستی \footnote{\lr{Biological-context-aware-based methods}}:}\\
اگرچه روش‌های بر پایه شبکه عملکرد خوبی دارند، اما عملکرد آنها می‌تواند با به کارگیری اطلاعات تکمیلی بهبود یابد. این اطلاعات می‌توانند از منابع گوناگونی مثل اطلاعات دامنه‌ای پروتئین‌ها، برچسب‌های ژن شناسی، نمایه بیان ژنی جمع آوری شوند. پژوهش آلن و همکارانش \cite{2013PCIA}، الگوریتم \lr{PCIA} را توسعه داده‌اند که از ترکیب اطلاعات \lr{GO} در کنار ساختار شبکه استفاده می‌کند.  پژوهش دیگر ژانگ و همکارانش \cite{2019integrating} رابطه‌ی بین شکل گیری مجموعه‌های پروتئینی و هم بیانی پروتئین‌ها را نشان داده است.
\begin{itemize}
    \item روش‌های هسته-اتصال \footnote{\lr{Core-attachment}}: روش‌های هسته-اتصال بر پایه این ایده هستند که هر مجموعه پروتئینی از یک هسته تشکیل شده‌ است که شامل پروتئین‌هایی با هم بیانی بالا می‌باشند. الگوریتم \lr{COACH} \cite{2009core} یکی از شناخته شده‌ترین الگوریتم‌های این دسته است که از دو مرحله شناسایی پروتئین‌های هسته‌ای و اضافه کردن پروتئین‌ها به پروتئین‌های هسته‌ای تشکیل شده است. تمرکز این الگوریتم بر ایجاد مجموعه‌های پروتئینی است که از نظر زیستی نیز با معنی باشند. الگوریتم \lr{CORE} \cite{2009predicting} نیز از سه مرحله‌، پیش‌بینی پروتئین‌های هسته‌ای، حذف هسته‌های با اهمیت پایین (بر اساس یک معیار اتصال)، و محاسبه اهمیت مجموعه‌های شناسایی شده، تشکیل شده است. اخیرا نیز الگوریتم \lr{CO-DPC} از این دسته ‌بندی ارائه شده است که از نمایه بیان ژنی در کنار شبکه \lr{PPI} استفاده می‌کند.
    \item الگوریتم‌های مبتنی بر اطلاعات عملکردی \footnote{\lr{Functional-information-based }}: دسته دوم الگوریتم‌ها روش‌های مبتنی بر اطلاعات عملکردی هستند که از اطلاعات ناهمگون پروتئین‌ها به منظور شناسایی مجموعه‌های با معنی استفاده می‌کنند. یکی از الگوریتم‌های این دسته، الگوریتم \lr{PCP} \cite{2008PCP} است که از اطلاعات ساختاری به منظور وزن‌دهی شبکه \lr{PPI} استفاده می‌کند. سپس ابتدا اقدام به شناسایی کلیک‌های بیشینه\footnote{\lr{Maximal clique}}  در شبکه \lr{PPI} کرده، در مرحله بعد چگالی بین خوشه‌ها را محاسبه می‌کند و در نهایت اقدام به ترکیب جزئی کلیک‌ها می‌کند.
\end{itemize}

لازم به ذکر استراتژی‌های دیگری که در سایر پژوهش‌های مربوط به پردازش گراف‌ها و خوشه‌بندی آنها خوب عمل کرده‌اند نیز مورد توجه قرار گرفته‌اند که از جمله آنها می‌توان به روش‌های بر پایه امبدینگ \cite{2019hierarchical}\cite{2018embeding} و تجزیه ماتریسی \cite{2020integrative} اشاره کرد که به منظور شناسایی مجموعه‌های پروتئینی نیز مورد استفاده قرار گرفته‌اند.
