\chapter{بررسی منابع}
در این قسمت به بررسی پژوهش‌های پیشینی که به منظور پیدا کردن مجموعه‌های پروتئینی در شبکه‌های \lr{PPI} انجام شده‌اند، می‌پردازیم. همانطور که در بخش‌های پیشین بررسی شد، تمرکز این پژوهش بر روی دید گرافی به شبکه‌های \lr{PPI} و ادغام اطلاعات زیست‌شناسی پروتئین‌ها به منظور تشخیص دقیق‌تر مجموعه‌های پروتئینی است. از آنجایی که پیدا کردن مجموعه‌های پروتئینی در شبکه‌های \lr{PPI} معادل خوشه‌بندی این شبکه‌ها می‌باشد، ما ابتدا چند نمونه از پژوهش‌های مرتبط با خوشه‌بندی گراف‌های دارای گره‌ ویژگی که بیشترین ارتباط را با هدف پژوهش ما دارند را بررسی می‌کنیم. 
\section{خوشه‌بندی گراف‌های دارای گره ویژگی}
پژوهش وحید جان‌نثاری و همکارانش\cite{nonnegative} ، یک الگوریتم بر پایه تجزیه نامنفی ماتریسی\LTRfootnote{\lr{Non-negative matrix factorization}}  به منظور خوشه‌بندی گراف‌های ویژگی‌دار معرفی می‌کند. روش آن‌ها ابتدا اطلاعات ساختاری که توسط ماتریس همسایگی\LTRfootnote{\lr{Adjacency matrix}}  نشان داده می‌شود را به کمک تجزیه نامنفی متقارن ماتریس\LTRfootnote{\lr{Symmetric non-negative matrix factorization}}  و اطلاعات ویژگی‌های گره‌ها را به کمک تجزیه نامنفی بازتابی ماتریس\LTRfootnote{\lr{Projective non-negative matrix factorization}}  به یک فضای کم بعد مختص خوشه‌بندی (هم بعد با تعداد خوشه‌ها) به صورت جداگانه انتقال می‌دهد که درجه عضویت هر گره به هر خوشه را نمایش می‌دهد. همینطور به منظور حفظ ثبات در خوشه بندی در هر دو فضا اقدام به نزدیک کردن این دو ماتریس به کمک تابع هدف می‌کند که به صورت مقابل تعریف شده است:
\begin{gather}
    J_{of}=min||A-VV^{T}||_{F}^{2}+\alpha||VV^{T}-UU^{T}||_{F}^{2}+||F-UU^{T}F||_{F}^2 \\
    s.t.\:\: V\ge 0,\: U\ge 0,\:V^TV=I,\: U^TU=I.\nonumber  
\end{gather}

که در تابع هدف، \lr{A} ماتریس همسایگی، $V\in R^{n\times k}$ ماتریس حاصل از تجزیه نامنفی متقارن ماتریس \lr{A} است. همینطور با در نظر  گرفتن \lr{M} (ماتریس شباهت\LTRfootnote{\lr{Similarity matrix}}  گره‌ها براساس ماتریس ویژگی‌ها) به صورت $M=UU^T;U\in R^{n\times k}$ و عبارت سوم در بهینه سازی که به صورت مقابل بیان شده است:  $|F-MF|$ در واقع اقدام به استفاده از ویژگی خودبیانگری\LTRfootnote{\lr{Self-expression }}  داده‌ها کرده‌اند، که در نتیجه روش بیان شده را می‌توان یک روش ترکیبی از خوشه‌بندی زیر فضا\LTRfootnote{\lr{Subspace clustering}}  و تجزیه نامنفی ماتریس در نظر گرفت.

در پژوهشی دیگر توسط کانگ و همکارانش\cite{fine} ، یک روش بر پایه شبکه‌های پیچشی گرافی\LTRfootnote{\lr{Graph convolutional networks}}  و خوشه‌بندی طیفی ارائه شده است. ایده اصلی در این روش بر پایه پردازش سیگنالی گراف است که در آن یک فیلتر پایین گذر\LTRfootnote{\lr{Low-pass filter}}  را به منظور نزدیک کردن و ادغام ویژگی‌های گره‌ها و ساختار گراف به ماتریس ویژگی‌ها اعمال می‌کنند. در نتیجه یک بازنمایی جدید بر این اساس را برای گره‌ها بدست می‌آورند:
\begin{equation}
\bar{X}=(I-1/2 L)^k X    
\end{equation}

همچنین در فرمول بالا \lr{k} یک هایپر پارامتر است که میزان مرتبه مجاورت بازنمایی به دست آمده را مشخص می‌کند به عبارت دیگر مقادیر کوچک‌تر \lr{k} دید محلی تری به ساختار گراف دارند و بالعکس. \lr{L}  ماتریس لاپلاسی نرمال شده\LTRfootnote{\lr{Normalized Laplacian matrix}} است که به صورت $L = I - A$ تعریف می‌شود.
 در مرحله بعد برای اعمال خوشه‌بندی طیفی، نیاز به محاسبه ماتریس شباهت بین گره‌ها است که به صورت مقابل عمل کرده‌اند.
\vspace{-1mm}
\begin{equation}
\min_S||\bar{X} ^T- \bar{X} ^T S||_F^2+α||S-f(A)||_F^2    
\end{equation}


که در اینجا ماتریس شباهت \lr{S} از بهینه سازی تابع هدف بالا بدست می‌آید و سپس با یک انتقال به یک ماتریس متقارن نامنفی تبدیل شده و در نهایت نیز خوشه‌بندی طیفی روی آن اعمال می‌شود. یکی از مشکلات این روش انتخاب مناسب هایپر پارامتر \lr{K} است که به طور مستقیم بر خروجی الگوریتم تاثیر می‌گذارد که توسط پژوهش دیگری که توسط ژانگ و همکارانش\cite{adaptive}  انجام شده‌ است، دو استراتژی \lr{AGC} و \lr{IAGC} برای پیدا کردن مقدار مناسب \lr{k} ارائه شده است.

بهومیک و همکاران \cite{bhowmick2024dgcluster} روشی مبتنی بر بیشینه‌سازی ماژولاریتی با عنوان \lr{DGCluster} ارائه داده‌اند که در آن از شبکه‌های عصبی گرافی برای خوشه‌بندی گراف‌های ویژگی‌محور استفاده می‌شود. در این چارچوب، ابتدا یک شبکه عصبی گرافی پیچشی دو‌لایه با تابع فعال‌ساز \lr{SELU} به‌منظور استخراج بردارهای تعبیه گره‌ها به‌کار گرفته می‌شود. سپس با اعمال مجموعه‌ای از تبدیلات غیرخطی و نرمال‌سازی، تعبیه‌های حاصل به فضای مختصات مثبت محدود می‌گردند تا محاسبه شباهت بین گره‌ها تسهیل شود. در ادامه، با استفاده از شباهت کسینوسی میان بردارهای تعبیه گره‌ها، نسخه‌ تغییر یافته‌ای از ماژولاریتی تعریف شده و تابع هزینه متناظر با آن به‌صورت منفی ماژولاریتی معادله‌بندی می‌شود.
\begin{equation}
	L_1 = - \hat{Q} = \frac{1}{2m}Tr(BXX^{T})
\end{equation}
که در آن هر درایه ماتریس $B$ به صورت $B_{ij} = (A_{ij} - \frac{d_i d_j}{2m})$ و $X$ ماتریس تعبیه هر گره حاصل از شبکه عصبی گرافی بعد می‌باشد.
پس از آموزش شبکه عصبی با هدف بیشینه‌سازی ماژولاریتی، فرآیند خوشه‌بندی نهایی بر روی بردارهای تعبیه آموخته‌شده و با استفاده از الگوریتم خوشه‌بندی سلسله‌مراتبی \lr{BIRCH} انجام می‌گیردکه نیازی به تعیین تعداد خوشه‌ها از قبل ندارد \cite{zhang1996birch}.

در پژوهش هی و همکاران \cite{he2024detecting}، یک چارچوب یکپارچه با عنوان \lr{SSAGCN}\LTRfootnote{Self-Supervised Adaptive Graph Convolutional Network} برای خوشه‌بندی گراف‌ها معرفی شده است. در این روش، ابتدا گراف ورودی ویژگی‌دار به‌صورت $G(V,E,A,X)$ در نظر گرفته می‌شود. سپس گراف کمکی دیگری به‌شکل $G(V,E_a,A_a,X)$ ساخته می‌شود که تفاوت آن با گراف اولیه در مجموعه یال‌ها و ماتریس مجاورت است. این گراف کمکی بر اساس محاسبه شباهت بین ویژگی‌های گره‌ها با استفاده از معیار شباهت کسینوسی تشکیل شده و در آن هر گره به $k$ گره مشابه خود متصل می‌شود. در ادامه، هر دو گراف به دو شبکه عصبی گرافی پیچشی با وزن‌های به‌اشتراک‌گذاشته‌شده وارد شده و بردارهای تعبیه حاصل از آن‌ها با بهره‌گیری از یک مکانیزم توجه برای هر گره ترکیب می‌شوند. تابع هزینه پیشنهادی به‌صورت ترکیب خطی و مبتنی بر بیشینه‌سازی ماژولاریتی، بازسازی ماتریس مجاورت و بازسازی ماتریس ویژگی‌ها تعریف شده است که از آن با عنوان مکانیزم رمزگشای دوگانه\LTRfootnote{Dual Decoder} یاد می‌شود. در نهایت، با استفاده از تعبیه‌های نهایی به‌دست‌آمده، میزان تعلق هر گره به خوشه‌های مختلف محاسبه شده و هر گره به خوشه‌ای اختصاص می‌یابد که بیشترین میزان تعلق را داشته باشد.


یکی از مشکلات روش‌های بر پایه بازنمایی این است که دو فرآیند بازنمایی‌ داده‌ها و خوشه‌بندی از یکدیگر مستقل‌اند در نتیجه نمی‌توان اطمینان داشت که بازنمایی‌های ایجاد شده برای وظیفه مورد‌نظر (در اینجا خوشه‌بندی) مناسب هستند و همچنین نمی‌توان الگوریتم بازنمایی را بر اساس خطای خوشه‌بندی به طور مناسب به روزرسانی نمود. از این روی، وانگ و همکاران\cite{2019attributed}  یک روش خوشه بندی یکپارچه توجه محور بر پایه شبکه عصبی گراف ارائه داده‌اند که مرحله بازنمایی و خوشه‌بندی را با هم ترکیب می‌کند. در این پژوهش از یک شبکه گرافی توجه محور\LTRfootnote{\lr{Graph attentional}}  به عنوان کدگذار  استفاده شده‌ است. ضرایب توجه کدگذار\LTRfootnote{\lr{Decoder}} با استفاده از یک ماتریس مجاورت با مرتبه بالا همانند پژوهش قبلی محاسبه می‌شوند. قسمت کدگشا\LTRfootnote{\lr{Encoder}}  نیز از ضرب داخلی بردار‌های بازنمایی کدگذار به منظور بازسازی ماتریس مجاورت گراف استفاده می‌کند که برای خروجی این قسمت تابع هزینه بازسازی در نظر گرفته شده است. نوآوری این مقاله در معرفی مفهوم بازنمایی خود بهینه‌ساز است که در آن به طور مکرر نقاط مربوط به هر خوشه براساس مقدار اطمینان تعلق به خوشه به‌روزرسانی می‌شود و به طور همزمان بازنمایی‌ها را نیز به وسیله آن اصلاح می‌کند.

%% 2024-DGCLUSTER A Neural Framework for Attributed Graph Clustering via...


%% 2024-Detecting communities with multiple topics in attributed networks via self-supervised adaptive graph convolutional network - 3158ad0100491b03cc2923fe27f95183
%% 2019-Overlapping Community Detection with Graph Neural Networks
از جمله پژوهش‌های شاخص در زمینه خوشه‌بندی گراف‌های ویژگی‌محور\LTRfootnote{Attributed Graphs} می‌توان به کار شچور\LTRfootnote{Shchur} و گانمن\LTRfootnote{Günnemann} اشاره نمود \cite{nocd2019} که به‌عنوان مبنای اصلی این پایان‌نامه نیز مورد استفاده قرار گرفته است. در این پژوهش، یک الگوریتم یکپارچه برای شناسایی خوشه‌های هم‌پوشان در گراف‌ها ارائه شده است که ایده اصلی آن بر تلفیق توانمندی‌های شبکه‌های عصبی گرافی با یک مدل مولد احتمالی برنولی-پواسون\LTRfootnote{Probabilistic Bernoulli–Poisson Generative Model} استوار است.

در چارچوب این روش، گراف ورودی به‌صورت $G(A, X)$ در نظر گرفته می‌شود که در آن ماتریس مجاورت به صورت $A \in \{0,1\}^{n \times n}$ و $X \in \mathbb{R}^{n \times d}$ ماتریس ویژگی گره‌ها است. هدف، آموزش یک شبکه عصبی گرافی با پارامترهای $\theta$، موسوم به $GNN_{\theta}$، به‌منظور استخراج ماتریس عضویت خوشه‌ای $F$ می‌باشد:
\begin{equation}
	F = GNN_{\theta}(A, X)
\end{equation}
در این رابطه، $F \in \mathbb{R}{\ge 0}^{n \times C}$ ماتریس انتساب گره‌ها به خوشه‌ها بوده و هر مؤلفه $F_{uc}$ بیانگر میزان یا شدت تعلق گره $u$ به خوشه $c$ است. شبکه عصبی گرافی به‌کاررفته در این پژوهش، یک شبکه گرافی پیچشی دو‌لایه با تابع فعال‌ساز \lr{ReLU} است که مطابق معادله \ref{eq:gnn-nocd} تعریف می‌شود. در این ساختار، با افزودن یال‌های خوداتصالی و نرمال‌سازی ماتریس مجاورت، اطلاعات ساختاری و ویژگی‌های گره‌ها به‌صورت همزمان در فرآیند یادگیری مورد استفاده قرار می‌گیرند.
\begin{equation}
	\begin{aligned}
		& \tilde{A} = A + I_N \\
		& \tilde{D}_{ii} = \sum_{j} \tilde{A}_{ij} \\
		& \hat{A} = \tilde{D}^{-1/2}\tilde{A}\tilde{D}^{-1/2} \\
		& F=ReLU(\hat{A}ReLU(\hat{A}XW^{(1)})W^{(2)})
	\end{aligned}
	\label{eq:gnn-nocd}
\end{equation}
پس از استخراج ماتریس $F$، یک مدل مولد برنولی-پواسون به‌عنوان رمزگشا به‌کار گرفته می‌شود که هدف آن بازسازی ماتریس مجاورت گراف است. در این مدل، وجود یال میان دو گره $u$ و $v$ به‌صورت یک متغیر تصادفی برنولی با پارامتر $1-e^{-F_uF_v^T}$ مدل‌سازی می‌شود:
\begin{equation}
	A_{uv} \sim Bernoulli(1-e^{-F_uF_v^T})
\end{equation}
بر این اساس، احتمال مشاهده ماتریس مجاورت $A$ با شرط دانستن ماتریس عضویت $F$ به‌ صورت ضرب احتمال‌ها بر روی یال‌های موجود و ناموجود محاسبه می‌شود:
\begin{equation}
	P(A|F) = \prod_{A_{uv}\in E} 1-e^{-F_uF_v^T} \times \prod_{A_{uv}\notin E} e^{-F_uF_v^T}
\end{equation}
در نهایت، تابع هزینه مدل به‌ صورت $-log^{p(A|F)}$ تعریف شده و به شکل زیر قابل بیان است:
\begin{equation}
	L(F)=-\sum_{A_{uv}\in E}
	​log(1-exp(-F_uF_v^T))+\sum_{A_{uv}\notin E}
	​F_uF_v^T
\end{equation}
پژوهشگران در این مقاله به یک چالش اساسی در تابع هزینه فوق اشاره می‌کنند. از آن‌جا که گراف‌های واقعی عموماً تنک\LTRfootnote{Sparse} هستند، تعداد جفت‌گره‌هایی که بین آن‌ها یال وجود ندارد به‌مراتب بیشتر از یال‌های موجود است؛ در نتیجه، عبارت دوم تابع هزینه غالب شده و می‌تواند فرآیند یادگیری را تحت تأثیر قرار دهد. به‌منظور رفع این مشکل، نویسندگان با استفاده از امید ریاضی هر یک از عبارات تابع هزینه و فرض یک توزیع یکنواخت بر روی یال‌ها، نسخه اصلاح‌شده‌ای از تابع هزینه را ارائه می‌دهند که منجر به پایداری بیشتر و بهبود عملکرد مدل در گراف‌های تنک می‌شود.
\begin{equation}
	L(F)=-E_{(U,V)\sim P_E}[log(1-exp(-F_uF_v^T))]+E_{(u,v)\sim P_N}[F_uF_v^T]
\end{equation}
که در این معادله عبارت $P_E$ توزیع یکنواخت بر روی یال‌ها و $P_N$ توزیع یکنواخت بر روی گره‌هایی است که یالی بین آن‌ها وجود ندارد. پس از آموزش شبکه عصبی، برای پیدا کردن خوشه‌ها با استفاده از ماتریس وابستگی $F$ از آستانه $\varphi$ استفاده می‌شود.
\begin{equation}
	F_{uc} = 
	\begin{cases}
		1 & \text{if} \:\: F_{uc}>\varphi \\
		0 & \text{otherwise} 
	\end{cases}
\end{equation}
%% 
\section{پیش‌بینی مجموعه‌های پروتئینی}

در ادامه به بررسی روش‌های استفاده شده به منظور پیش‌بینی مجموعه‌های پروتئینی در شبکه‌های \lr{PPI} می‌پردازیم و یک دسته‌بندی برای این روش‌ها ارائه می‌دهیم.

\subsection{روش‌های بر پایه شبکه}\\
این روش‌ها تنها بر ساختار شبکه \lr{PPI} تمرکز می‌کنند. که به دو زیر دسته تقسیم می‌شوند:
\begin{itemize}
    \item روش‌های تقسیمی \LTRfootnote{\lr{Divisive methods }}: این دسته از روش‌ها، شبکه را به زیر شبکه‌ها تقسیم می‌کنند و این عمل را تا رسیدن به درجه دلخواه خوشه بندی تکرار می‌کنند. معروف ترین الگوریتم‌ این دسته الگوریتم خوشه‌بندی مارکوف\LTRfootnote{\lr{Markov clustering algorithm}}  \cite{2012markov} است که زیر شبکه‌ها را به کمک ولگشت تصادفی\LTRfootnote{\lr{Random walk}}  در شبکه پیدا می‌کند.
    \item روش‌های تجمعی \LTRfootnote{\lr{Agglomerative methods}}: با مجموعه کوچکی از پروتئین‌ها شروع کرده و با ترکیب آن‌ها اقدام به پیدا کردن مجموعه‌های پروتئینی نهایی می‌کند. الگوریتم \lr{CPNM} \cite{2020motif} یکی از الگوریتم‌‌های این دسته است که از تعبیه موتیف‌های\LTRfootnote{\lr{Motif}}  شبکه به منظور پیدا کردن نقش پروتئین‌ها استفاده می‌کند. سپس به منظور ایجاد بردار ویژگی‌ پروتئین‌ها از آن‌ها استفاده می‌شود. در نهایت نیز از روش پیدا کردن همسایگان به منظور شناسایی مجموعه‌های پروتئینی استفاده می‌کند. یکی دیگر از الگوریتم‌های تجمعی معروف الگوریتم \lr{ClusterONE} \cite{2012clusterone} است. این الگوریتم ابتدا پروتئین‌های با درجه‌ بالاتر را به عنوان پروتین‌های هسته \LTRfootnote{\lr{Seed}} (پروتین‌های آغازین)‌ در نظر می‌گیرد. سپس زیرگروه‌هایی از گره‌ها با بیشترین انسجام برای گره‌های هسته انتخاب می‌شوند. در انتها نیز گره هسته از بین گره‌هایی که مربوط به یک ترکیب شناخته شده نیستند انتخاب می‌شوند و این مراحل تکرار می‌شوند تا همه پروتئین‌ها به یک ترکیب مرتبط شوند. الگوریتم دیگر، \lr{MCODE}\LTRfootnote{\lr Molecular complex detection}  \cite{2003mcode} است که در سه مرحله انجام می‌شود. این الگوریتم ابتدا گره‌ها را وزن دهی می‌کند، سپس به شناسایی مجموعه‌ها می‌پردازد و در انتها نیز اقدام به اضافه / حذف کردن پروتئين‌ها به/از مجموعه‌های شناسایی شده با توجه به یک معیار اتصال می‌کند.
\end{itemize}

\subsection{روش‌های مبتنی بر آگاهی از زمینه‌های زیستی}\\
اگرچه روش‌های بر پایه شبکه عملکرد خوبی دارند، اما عملکرد آنها می‌تواند با به کارگیری اطلاعات تکمیلی بهبود یابد. این اطلاعات می‌توانند از منابع گوناگونی مثل اطلاعات دامنه‌ای پروتئین‌ها، برچسب‌های ژن شناسی، نمایه بیان ژنی جمع آوری شوند. پژوهش آلن و همکارانش \cite{2013PCIA}، الگوریتم \lr{PCIA} را توسعه داده‌اند که از ترکیب اطلاعات \lr{GO} در کنار ساختار شبکه استفاده می‌کند.  پژوهش دیگر ژانگ و همکارانش \cite{2019integrating} رابطه‌ی بین شکل گیری مجموعه‌های پروتئینی و هم بیانی پروتئین‌ها را نشان داده است.
\begin{itemize}
    \item روش‌های هسته-اتصال \LTRfootnote{\lr{Core-attachment}}: روش‌های هسته-اتصال بر پایه این ایده هستند که هر مجموعه پروتئینی از یک هسته تشکیل شده‌ است که شامل پروتئین‌هایی با هم بیانی بالا می‌باشند. الگوریتم \lr{COACH} \cite{2009core} یکی از شناخته شده‌ترین الگوریتم‌های این دسته است که از دو مرحله شناسایی پروتئین‌های هسته‌ای و اضافه کردن پروتئین‌ها به پروتئین‌های هسته‌ای تشکیل شده است. تمرکز این الگوریتم بر ایجاد مجموعه‌های پروتئینی است که از نظر زیستی نیز با معنی باشند. الگوریتم \lr{CORE} \cite{2009predicting} نیز از سه مرحله‌، پیش‌بینی پروتئین‌های هسته‌ای، حذف هسته‌های با اهمیت پایین (بر اساس یک معیار اتصال)، و محاسبه اهمیت مجموعه‌های شناسایی شده، تشکیل شده است. اخیرا نیز الگوریتم \lr{CO-DPC} از این دسته ‌بندی ارائه شده است که از نمایه بیان ژنی در کنار شبکه \lr{PPI} استفاده می‌کند.
    \item الگوریتم‌های مبتنی بر اطلاعات عملکردی \LTRfootnote{\lr{Functional-information-based }}: دسته دوم الگوریتم‌ها روش‌های مبتنی بر اطلاعات عملکردی هستند که از اطلاعات ناهمگون پروتئین‌ها به منظور شناسایی مجموعه‌های با معنی استفاده می‌کنند. یکی از الگوریتم‌های این دسته، الگوریتم \lr{PCP} \cite{2008PCP} است که از اطلاعات ساختاری به منظور وزن‌دهی شبکه \lr{PPI} استفاده می‌کند. سپس ابتدا اقدام به شناسایی کلیک‌های بیشینه\LTRfootnote{\lr{Maximal clique}}  در شبکه \lr{PPI} کرده، در مرحله بعد چگالی بین خوشه‌ها را محاسبه می‌کند و در نهایت اقدام به ترکیب جزئی کلیک‌ها می‌کند. از دیگر پژوهش‌‌های شاخص در این بخش می‌توان به پژوهش برهمند و همکارانش\cite{berahmand2021spectral} اشاره کرد. این پژوهش با تأکید بر این نکته که شناسایی دقیق‌تر ماژول‌های عملکردی مستلزم بهره‌گیری همزمان از ساختار شبکه و ویژگی‌های زیستی پروتئین‌هاست، روشی با عنوان \lr{TADW-SC} را پیشنهاد کرده‌اند. در این روش، علاوه بر ساختار شبکه \lr{PPI}، از ویژگی‌های استخراج‌شده از \lr{GO} نیز استفاده می‌شود. بدین منظور، در گام نخست با بهره‌گیری از روش \lr{TADW}\LTRfootnote{Text Associated Deep Walk}\cite{yang2015network}
    ، بردارهای تعبیه‌ای برای هر گره محاسبه می‌شوند که به‌گونه‌ای طراحی شده‌اند تا همزمان اطلاعات ساختاری گراف و ویژگی‌های گره‌ها را حفظ کنند. در مرحله بعد، با استفاده از بردارهای تعبیه آموخته‌شده، یک ماتریس شباهت بین گره‌ها ایجاد می‌شود. سپس الگوریتم خوشه‌بندی طیفی بر روی این ماتریس شباهت اعمال شده و ماژول‌های عملکردی استخراج می‌گردند. نتایج تجربی نشان می‌دهد که روش \lr{TADW-SC} با وجود ساختار ساده، در بسیاری از شبکه‌های \lr{PPI} عملکرد قابل قبولی از خود نشان داده و می‌تواند به‌عنوان یک رویکرد کارا برای شناسایی ماژول‌های عملکردی مورد استفاده قرار گیرد.
   
\end{itemize}


لازم به ذکر است که استراتژی‌های دیگری که در سایر پژوهش‌های مربوط به پردازش گراف‌ها و خوشه‌بندی آنها خوب عمل کرده‌اند نیز مورد توجه قرار گرفته‌اند که از جمله آنها می‌توان به روش‌های بر پایه تعبیه \cite{2019hierarchical}\cite{2018embeding} و تجزیه ماتریسی \cite{2020integrative} اشاره کرد که به منظور شناسایی مجموعه‌های پروتئینی نیز مورد استفاده قرار گرفته‌اند. 
\subsection{روش‌های مبتنی بر شبکه‌های عصبی گرافی}
در این میان، روش‌های مبتنی بر شبکه‌های عصبی گرافی برای شناسایی مجموعه‌های پروتئینی به‌طور محدود مورد مطالعه قرار گرفته‌اند و بررسی و توسعه این دسته از روش‌ها یکی از اهداف اصلی پژوهش حاضر به‌شمار می‌رود. با این حال، نتایج امیدوارکننده‌ای از به‌کارگیری شبکه‌های عصبی گرافی در این حوزه گزارش شده است؛ به‌گونه‌ای که روش پیشنهادی چن و همکاران \cite{chen2023adappi} در مقایسه با بسیاری از رویکردهای پیشین، بهترین عملکرد را در شناسایی ماژول‌های عملکردی شبکه‌های \lr{PPI} نشان داده است.
در پژوهش چن و همکاران، چارچوبی با عنوان \lr{AdaPPI} را برای شناسایی مجموعه‌های پروتئینی در شبکه‌های برهم‌کنش \lr{PPI} معرفی می‌کند. این چارچوب ترکیبی از یک شبکه عصبی گرافی پیچشی و یک راهبرد یادگیری تعبیه تطبیقی در سطح گره است که با هدف استخراج تعبیه‌هایی متناسب با ساختار محلی و سراسری شبکه طراحی شده است. در این روش، پس از استخراج تعبیه‌های گره‌ها، شناسایی ماژول‌های عملکردی با بهره‌گیری از تعبیه‌های آموخته‌شده و با استفاده از الگوریتم کلاسیک یافتن کلیک‌های هسته‌ای و گسترش آن‌ها انجام می‌شود.

در \lr{AdaPPI}، ویژگی‌های مربوط به \lr{GO} هر پروتئین شامل فرآیندهای زیستی و عملکردهای مولکولی، از زیرگراف‌های متناظر استخراج شده و به‌صورت بردارهای دودویی کدگذاری می‌شوند. این ویژگی‌ها به‌عنوان ورودی اولیه به شبکه عصبی گرافی پیچشی داده می‌شوند. در ادامه، خروجی هر لایه از شبکه عصبی گرافی به یک شبکه عصبی بازگشتی وارد می‌شود که وظیفه آن پیش‌بینی میزان هموارسازی مناسب برای تعبیه هر گره است. این شبکه بازگشتی \LTRfootnote{Recurrent Neural Network - RNN} به‌صورت همزمان با شبکه عصبی گرافی آموزش داده می‌شود.

ایده اصلی این سازوکار بر این فرض استوار است که گره‌های مختلف در شبکه PPI به سطوح متفاوتی از اطلاعات همسایگی نیاز دارند. به‌طور خاص، گره‌های با درجه بالا عمدتاً از اطلاعات همسایگان مرتبه پایین‌تر بهره می‌برند، در حالی که گره‌های با درجه پایین‌تر برای به‌روزرسانی تعبیه خود نیازمند دسترسی به اطلاعات همسایگان با مراتب بالاتر هستند. راهبرد یادگیری تعبیه تطبیقی در \lr{AdaPPI} از بروز پدیده هموارسازی بیش از حد یا کمتر از حد تعبیه‌ها جلوگیری کرده و امکان استخراج تعبیه‌های گره‌ای متمایزتر را فراهم می‌سازد.
پس از یادگیری تعبیه‌ها، با استفاده از ساختار گراف و تعبیه‌های حاصل، کلیک‌های هسته‌ای در شبکه شناسایی شده و سپس از طریق الگوریتم گسترش کلیک، ماژول‌های عملکردی نهایی استخراج می‌شوند. تابع هزینه مورد استفاده در این پژوهش به‌گونه‌ای طراحی شده است که فاصله بین تعبیه گره‌های متعلق به یک خوشه را کاهش داده و همزمان فاصله تعبیه گره‌های متعلق به خوشه‌های متفاوت را افزایش دهد. این تابع هزینه به‌صورت معادله \ref{eq:adappi-loss} تعریف می‌شود:
\begin{equation}
	\begin{aligned}
		L_{\text{intra}} &= \frac{1}{|C|} \sum_{c \in C} \frac{1}{|c|(|c|-1)} \sum_{i,j \in c,\, i \neq j} \lVert \bar{X}_i - \bar{X}_j \rVert_2, \\
		L_{\text{inter}} &= \frac{1}{|C|} \sum_{c \in C} \frac{1}{|c|(n-|c|)} \sum_{i \in c,\, j \notin c} \lVert \bar{X}_i - \bar{X}_j \rVert_2, \\
		L &= L_{\text{intra}} + \beta \frac{1}{L_{\text{inter}}}.
	\end{aligned}
	\label{eq:adappi-loss}
\end{equation}

با توجه به اینکه چارچوب پیشنهادی \lr{AdaPPI} از ترکیب شبکه‌های عصبی و الگوریتم‌های کلاسیک خوشه‌بندی بهره می‌برد و فرآیند شناسایی ماژول‌های عملکردی در دو مرحله مجزا انجام می‌شود، این روش در دسته الگوریتم‌های یکپارچه قرار نمی‌گیرد. با این حال، نتایج تجربی نشان می‌دهد که این روش در بسیاری از شبکه‌های PPI عملکرد بهتری نسبت به روش‌های پیشین داشته و به‌عنوان یکی از مؤثرترین الگوریتم‌ها برای شناسایی ماژول‌های عملکردی در این حوزه مطرح است.





