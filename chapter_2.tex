\chapter{مفاهیم بنیادی}

\section{مقدمه}
در این فصل، مفاهیم بنیادی و پیش‌نیازهایی که برای درک بهتر پژوهش حاضر ضروری هستند معرفی می‌شوند. از آنجا که این پایان‌نامه در تقاطع علوم زیستی و روش‌های محاسباتی قرار دارد، آشنایی با مفاهیم هر دو حوزه برای دنبال‌کردن مطالب فصل‌های بعدی اهمیت ویژه‌ای دارد.
بر همین اساس، در بخش نخست این فصل، مفاهیم زیستی مرتبط با موضوع پژوهش مورد بررسی قرار می‌گیرند. سپس در بخش دوم، به معرفی مفاهیم محاسباتی مورد استفاده پرداخته می‌شود و تمرکز اصلی بر مباحث مرتبط با شبکه‌های عصبی گرافی و چارچوب‌های یادگیری مبتنی بر گراف خواهد بود.
در نهایت، در بخش پایانی این فصل، معیارهای ارزیابی به‌کاررفته در این پژوهش برای سنجش کیفیت شناسایی ماژول‌های عملکردی معرفی شده و به‌طور خلاصه تشریح می‌شوند تا زمینه لازم برای تحلیل نتایج در فصل‌های بعدی فراهم شود.

\section{مفاهیم زیستی}
در این تحقیق یک روش محاسباتی به منظور شناسایی مجموعه‌های پروتئینی ارائه می‌شود که در حین آن، به برخی از مفاهیم زیستی برمی‌خوریم که در این قسمت توضیحات مختصر و مفیدی از هر یک جهت فهم راحت‌تر موضوع پژوهش ارائه شده است.
\subsection{پروتئین}
پروتئین‌ها از مهم‌ترین  درشت مولکول‌های زیستی در سلول‌های زنده به‌شمار می‌آیند که از تکرار واحدهای اسید آمینه ساخته شده‌اند و نقش‌های حیاتی و متنوعی را در ساختار و عملکرد سلول ایفا می‌کنند. این مولکول‌ها به‌طور ویژه به‌عنوان کارگزاران مولکولی سلول شناخته می‌شوند، زیرا در فرایندهایی مانند کاتالیز واکنش‌های شیمیایی (از طریق آنزیم‌ها)، انتقال مولکول‌ها، پیام‌رسانی درون‌سلولی، ایجاد حرکت و حفظ یکپارچگی ساختار سلولی نقش اساسی دارند. توالی اسیدهای آمینه هر پروتئین توسط ژن مربوطه تعیین شده و طی فرایند ترجمه از روی آران‌ای(ریبونوکلئیک اسید)\footnote{\lr{mRNA: messenger ribonucleic acid}} پیام‌رسان سنتز می‌شود. پس از سنتز، پروتئین‌ها از طریق فرایند تاخوردگی\footnote{Folding} به ساختارهای سه‌بعدی مشخصی دست می‌یابند که برای عملکرد زیستی آن‌ها ضروری است. ویژگی‌های عملکردی هر پروتئین به ترتیب خاص اسیدهای آمینه و برهم‌کنش‌های فضایی میان آن‌ها وابسته است. گستردگی و تنوع عملکردهای پروتئین‌ها به‌گونه‌ای است که تقریباً تمامی فرایندهای زیستی سلول، به‌صورت مستقیم یا غیرمستقیم، تحت تأثیر یا کنترل آن‌ها قرار دارند \cite{Alberts2022MBC}.

\subsection{ماژول‌های عملکردی}

فعالیت‌های زیستی در سلول و به‌طور کلی در بدن، معمولاً حاصل عملکرد یک پروتئین منفرد نیستند، بلکه نتیجه‌ی همکاری هماهنگ مجموعه‌ای از پروتئین‌ها می‌باشند که به‌صورت سازمان‌یافته با یکدیگر در ارتباط هستند. این پروتئین‌ها از طریق تعاملات مختلف، به‌ویژه تعاملات فیزیکی، در انجام یک یا چند وظیفه‌ی زیستی مشخص مشارکت می‌کنند \cite{adappi}.

به چنین مجموعه‌ای از پروتئین‌ها که به‌صورت هماهنگ برای انجام یک عملکرد زیستی مشترک عمل می‌کنند، مجموعه‌ی پروتئینی یا ماژول عملکردی\footnote{\lr{Functional Module}} گفته می‌شود. هر ماژول عملکردی معمولاً بیانگر یک فرآیند زیستی، مسیر مولکولی یا سازوکار تنظیمی خاص در سلول است و اجزای آن، از نظر عملکردی به یکدیگر وابسته‌اند \cite{complexRefLocal}.

تعامل فیزیکی میان پروتئین‌ها که تحت عنوان تعامل پروتئین‌--پروتئین شناخته می‌شود، نقش محوری در شکل‌گیری و پایداری این ماژول‌های عملکردی ایفا می‌کند. این تعاملات امکان انتقال سیگنال، تنظیم فعالیت‌های آنزیمی و هماهنگی زمانی و مکانی پروتئین‌ها را فراهم می‌سازند و از این رو، برای درک صحیح بسیاری از فعالیت‌های زیستی ضروری هستند \cite{hartwell1999molecular}.

از جمله فرآیندهای زیستی مهمی که مبتنی بر ماژول‌های عملکردی هستند می‌توان به رونوشت دی‌اِن‌اِی، رونوشت آر‌اِن‌اِی پیام‌رسان و تنظیم چرخه‌ی سلولی اشاره کرد. در هر یک از این فرآیندها، گروه مشخصی از پروتئین‌ها به‌صورت شبکه‌ای از تعاملات عمل می‌کنند و اختلال در هر یک از اجزای این شبکه می‌تواند منجر به بروز نقص عملکردی در کل فرآیند شود.

در سال‌های اخیر، پیشرفت در شناسایی و تحلیل ماژول‌های عملکردی، به یکی از موضوعات مهم در زیست‌شناسی سامانه‌ای و بیوانفورماتیک تبدیل شده است. شناسایی دقیق این ماژول‌ها کاربردهای گسترده‌ای از جمله پیش‌بینی عملکرد پروتئین‌های ناشناخته \cite{li2012towards}، درک مکانیسم‌های مولکولی بیماری‌ها \cite{safari2014protein} و کشف اهداف دارویی جدید \cite{mujawar2019delineating} دارد. از این رو، مطالعه و مدل‌سازی ماژول‌های عملکردی نقش کلیدی در توسعه روش‌های نوین تشخیصی و درمانی ایفا می‌کند.


\subsection{بیان ژن\footnote{\lr{Gene expression}}}
بیان ژن  فرآیندی است که طی آن اطلاعات نهفته در توالی دی‌اِن‌اِی به محصولات عملکردی، عمدتاً آر‌اِن‌اِی و پروتئین، تبدیل می‌شود. این فرآیند شامل مراحل متعددی از جمله رونوشت دی‌اِن‌اِی به آر‌اِن‌اِی و در بسیاری از موارد ترجمه آراِن‌اِی ‌به پروتئین است و نقش اساسی در تعیین ساختار، عملکرد و رفتار سلول ایفا می‌کند. سطح بیان هر ژن نشان‌دهنده‌ی میزان فعالیت آن ژن در یک شرایط زیستی خاص بوده و به‌طور دقیق تحت تأثیر سازوکارهای تنظیمی مختلفی مانند عوامل رونویسی، تغییرات اپی‌ژنتیکی و سیگنال‌های درون‌سلولی و برون‌سلولی قرار دارد. تفاوت در الگوهای بیان ژن میان سلول‌ها، بافت‌ها یا شرایط فیزیولوژیک و پاتولوژیک مختلف، عامل اصلی تنوع عملکردی سلول‌ها محسوب می‌شود. از این رو، تحلیل داده‌های بیان ژن ابزار مهمی برای درک فرآیندهای زیستی، شناسایی مسیرهای مولکولی مختل‌شده در بیماری‌ها و استخراج نشانگرهای زیستی به‌شمار می‌رود \cite{gelard2024bulkrnabert}.

\subsection{پایگاه داده هستی شناسی ژن\footnote{\lr{Gene Ontology}}}
\lr{GO} یک بانک داده و سیستم طبقه‌بندی است که با هدف ایجاد یک زبان استاندارد برای توصیف ژن‌ها و محصولات ژنی (که پروتئین‌ها نیز جزو آنها هستند) ایجاد شده است.
این پروژه اطلاعات ساختاریافته و قابل پرداش از فرايند‌های زیستی، عملکرد مولکولی و مولفه‌ی سلولی ژن‌ها فراهم می‌کند.
داده‌های پروژه \lr{GO} به صورت گسترده‌ای در تحقیقات مربوط به علوم زیستی مورد استفاده قرار می‌گیرد و همینطور همواره اطلاعات آن از نظر کمیت و کیفیت درحال تغییر است \cite{gene2019gene}.

هر عبارت GO شامل موارد زیر می‌شود \cite{goOverview}:
\begin{itemize}
    \item یک نام که برای انسان قابل فهم باشد.
    \item یک شناساگر مختص آن عبارت که با پیشوند \lr{GO:} آغاز می‌شود.
    \item یک تعریف مختصر از مفاهیمی که توسط این عبارت GO نمایش داده می‌شود.
    \item ارتباط آن با سایر عبارات \lr{GO}؛ که در گراف GO هر عبارت (به جز عبارات ریشه‌ای) فرزند یک عبارت GO دیگر است.
\end{itemize}

اطلاعات موجود در بانک داده \lr{GO} به صورت ساختمان داده گرافی ذخیره شده‌اند. هر عبارت داری یک یا چند فرزند است که در نتیجه ساختار گراف \lr{GO}، یک گراف جهت‌دار بدون دور\footnote{Directed acyclic graph} است .
گراف \lr{GO} شامل چهار نوع یال \lr{\texttt{is\_a}}، \lr{\texttt{part\_of}}، \lr{\texttt{regulates}} و \lr{\texttt{has\_part}} است که هر یک به‌ترتیب بیانگر رابطهٔ «نوعی از»، «جزئی از»، «نقش تنظیم‌کنندگی» و «دارا بودن جزء» میان مفاهیم مختلف در این هستی‌شناسی می‌باشند \cite{ashburner2000gene}.
این سیستم شامل سه زیرگراف جهت‌دار بدون دور اصلی است که هر یک از آنها  جنبه خاصی از عملکرد زیستی را توصیف می‌کنند:

\textbf{فرآیند زیستی\footnote{\lr{Biological process}}} : این بخش به فرآیند‌های زیستی اشاره دارد که ژن و یا پروتئین خاصی در آن نقش دارد.

\textbf{عملکرد مولکولی\footnote{\lr{Molecular function}}} : این بخش عملکرد دقیق مولکولی ژن یا پروتئین را توصیف می‌کند.

\textbf{مولفه‌ی سلولی\footnote{\lr{Cellular component}}} : این بخش به مکانی که ژن یا پروتئین در آن قرار دارد اشاره می‌کند.
از ویژگی‌های دیگر این بانک داده نمایش اطلاعات به صورت سازماندهی شده و سلسله مراتبی است که شامل شبکه‌های بدون دور می‌شود و ویژگی‌ها به این صورت مرتب شده‌اند \cite{gene}. 

\subsection{شبکه‌های PPI و ویژگی‌های آنها }
یک شبکه \lr{PPI} معمولا به صورت یک گراف بدون جهت $G=(V,E)$ نشان داده می شود که $V$ و $E$ به ترتیب نمایان‌گر پروتئین ها و تعاملات بین آنها می باشند. وزن های روی یال ها را می توان برای توصیف ویژگی های شبکه \lr{PPI}، مانند ویژگی‌های توپولوژیکی یا عملکردی استفاده کرد. شبکه های \lr{PPI} سه ویژگی توپولوژیکی زیر را دارند:

\begin{itemize}
    \item توزیع بدون مقیاس\footnote{\lr{Scale-free distribution}} : $P(k)$   مفهوم توزیع درجه یعنی احتمال اینکه یک گره در یک شبکه دقیقا \lr{k} پیوند داشته باشد را نشان می دهد. یک  شبکه \lr{PPI} دارای توزیع درجه توانی $P(k)\sim\:k^{-\lambda}$ می باشد\cite{evolution} . این ویژگی به این معنی است که پروتئین‌های تعامل‌دار در شبکه‌های \lr{PPI} به طور یکنواخت توزیع نمی شوند‌، بیشتر پروتئین‌ها تنها در چند تعامل شرکت می‌کنند در حالی که مجموعه کوچکی از پروتئین‌ها در ده‌ها تعامل (تشکیل گره هاب\footnote{\lr{Hub}}) شرکت می‌کنند. 
    \item ویژگی جهان کوچک\footnote{\lr{Small-world property}} : پروتئین‌های یک شبکه \lr{PPI} دارای میانگین طول مسیر کم و ضرایب خوشه‌ای بالا هستند\cite{protein}  که سیگنال‌های هر گره در شبکه \lr{PPI} را قادر می‌سازد تا از طریق چند جهش به سرعت به هر گره دیگری برسند. در نتیجه شبکه‌های \lr{PPI} هم زمان انتقال سیگنال و هم زمان پاسخ کوتاهی خواهند داشت.
    \item شبکه با ماژول‌های عملکردی\footnote{\lr{Functional modular network}} : شبکه \lr{PPI} یک شبکه ماژولار و سلسله مراتبی می‌باشد. یک ماژول عملکردی در یک شبکه \lr{PPI} یک مجموعه با بیشترین تعداد پروتئین که عملکرد یکسانی دارند، می‌باشد. بارزترین مشخصه ماژول عملکردی، ارتباط بین ساختار توپولوژیکی شبکه \lr{PPI} و عملکرد‌ پروتئین‌های آن است که مبنای بسیاری از روش‌های تشخیص ماژول عملکردی است\cite{molecular} \cite{road}.
\end{itemize}


\section{مفاهیم محاسباتی}

در مفاهیم بنیادی ابتدا به فرمول بندی مسئله خوشه‌بندی گراف‌های با گره‌های دارای ویژگی می‌پردازیم. 

\subsection{خوشه بندی در گراف های با گره های دارای ویژگی }
با فرض گراف $G=(V,E,F)$ که در آن $V$ مجموعه گره ها، $E$ مجموعه یال ها است و $F$ ماتریس ویژگی های گره ها می باشد، یک خوشه بندی از گراف $G$ را می توان با $C$ نشان داد که مجموعه ای از زیر مجموعه های $V$ است، به صورتی که  $C_i\in C\:;\:C_i\subset V$. هدف از خوشه بندی این است که خوشه هایی که هم از نظر ساختاری و هم از نظر ویژگی های گره‌ها بهم بیشترین شباهت را دارند، پیدا کنیم. همچنین خوشه‌های ایجاد شده باید از نظر ارتباط‌ یال‌های داخل خوشه چگال و در ارتباط یال‌ها با دیگر خوشه‌ها تنک باشند.

\subsection{دسته‌بندی و روش‌های کلی خوشه‌بندی گراف}
 روش‌های خوشه‌بندی گراف را می‌توان از دیدگاه‌های مختلفی تقسیم‌بندی کرد. این تقسیم‌بندی‌ها بر اساس معیارها و ویژگی‌های خاصی صورت می‌گیرند که به نحوه برخورد با داده‌های گرافی، نوع اطلاعات استفاده شده، و تکنیک‌های به کار گرفته شده بستگی دارد. در این پژوهش از آنجایی که نوع گراف ورودی مشخص است و قصد خوشه‌بندی گراف‌های \lr{PPI} با گره‌های دارای ویژگی را داریم، روش‌های خوشه‌ بندی را بر اساس روش‌ مورد استفاده تقسیم‌بندی می‌کنیم:

\begin{itemize}
    \item روش‌های طیفی\footnote{\lr{Spectral clustering}} : از مقادیر ویژه\footnote{\lr{Eigenvalues }}  ماتریس لاپلاسین یا مجاورت برای یافتن خوشه‌ها استفاده می‌کنند.
    \item روش‌های فاکتورگیری ماتریسی\footnote{\lr{Matrix factorization}} : از روش‌های تجزیه‌ ماتریسی مانند تجزیه نامنفی ماتریس\footnote{\lr{Non-negative matrix factorization}}  یا تجزیه مقدار تکین\footnote{\lr{Singular value factorization}}  برای ایجاد امبدینگ و خوشه‌بندی استفاده می‌کنند.
    \item روش‌های سلسله‌مراتبی\footnote{\lr{Hierarchical clustering}} : گراف را به صورت سلسله ‌مراتبی خوشه‌بندی می‌کنند که به دو روش تقسیمی و تجمعی دسته‌بندی می‌شوند.
    \item روش‌های مبتنی بر امبدینگ\footnote{\lr{Embedding-based methods}} : ابتدا گره‌ها به فضای برداری کم‌بعد نگاشت می‌شوند و سپس خوشه‌بندی روی این فضای برداری انجام می‌شود و تمرکز اصلی در این روش‌ها یافتن بازنمایی مناسب برای خوشه‌بندی گراف است. \lr{(Node2Vec, DeepWalk, GCN, GNN)}.
    \item روش‌های بدون امبدینگ\footnote{\lr{Non-embedding methods}} : مستقیماً از ساختار گراف برای خوشه‌بندی استفاده می‌شود بدون اینکه گره‌ها به فضای برداری منتقل شوند \lr{(Louvain, graph - cut based)}.

\end{itemize}

\section{معیار‌های ارزیابی}
در این قسمت به بررسی معیار‌های ارزیابی عملکرد الگوریتم‌های شناسایی مجموعه‌های پروتئینی می‌پردازیم. در بین معیار‌های موجود، معیارهای دقت\footnote{\lr{Precision}}، بازیابی\footnote{\lr{Recall}}، صحت\footnote{\lr{Accuracy}}، امتیاز F ، بیشترین استفاده را در بین پژوهش‌ها داشته‌اند که ما نیز به منظور تحلیل و مقایسه عملکرد روش خود از آنها استفاده می‌کنیم.
در ابتدا برای شروع به معیار شباهت همسایگی که برای محاسبه تمامی معیار‌های مذکور مورد نیاز است، می‌پردازیم:
\subsection{شباهت همسایگی\footnote{\lr{Neighborhood affinity}}}

با در نظر گرفتن $P$ به عنوان مجموعه‌ای از مجموعه‌های پروتئینی شناسایی شده توسط الگوریتم، عملکرد الگوریتم به وسیله تعداد مجموعه‌های پروتئینی مشترک بین $P$ و مجموعه‌ای از مجموعه پروتئین‌های مرجع\footnote{\lr{Reference protein complex}} $B$ بدست می‌آید. برای مشخص کردن اینکه آیا یک مجموعه پروتئین شناسایی شده $p \in P$ با یک مجموعه پروتئین مرجع $b \in B$ یکسان هستند یا خیر ما اقدام به محاسبه معیار شباهت همسایگی به صورت مقابل می‌کنیم:
\begin{equation}
NA(p,b) = \frac{|V_p \cap V_b|^2}{|V_p| \times |V_b|}
\end{equation}

که $V_p$ مجموعه پروتئین‌های حاضر در ترکیب $p$ و به طور مشابه $V_b$ مجموعه پروتئین‌های حاضر در $b$ هستند. برای تفسیر شباهت همسایگی یک آستانه\footnote{\lr{Threshold}} از قبل تعیین شده (معمولا 25/0) در نظر گرفته می‌شود که شباهت همسایگی‌های بالاتر از آستانه به معنی یکسانی دو مجموعه است. همچنین تعداد مجموعه‌های شناسایی شده‌ای که حداقل با یک مجموعه مرجع یکسان در نظر گرفته می‌شوند را با $N_{cp}$ و تعداد مجموعه های مرجعی که حداقل با یکی از مجموعه‌های شناسایی شده الگوریتمی یکسان در نظر گرفته می‌شوند را با $N_{cb}$ نمایش می‌دهیم \cite{spectral}.
\begin{equation}
N_{cp} = \{p | p \in P, \exists b \in B, NA(p,b) \ge \omega\}
\end{equation}

\begin{equation}
N_{cb} = \{b | b \in B, \exists p \in P, NA(p,b) \ge \omega\}
\end{equation}

\subsection{دقت}
دقت یک معیار ارزیابی مجموعه پروتئینی‌های شناسایی شده است که نشان می‌دهد چند مورد از مجموعه‌های پیش‌بینی شده الگوریتم به درستی انتخاب شده‌اند.
\begin{equation}
Precision = \frac{N_{cp}}{|P|}
\end{equation}

\subsection{بازیابی}
بازیابی دیگر معیار مورد توجه است که نشان می‌دهد چند مورد از مجموعه پروتئینی‌های مرجع توسط الگوریتم پیش‌بینی شده‌اند. به دیگر عبارت میزان پوشش الگوریتم از مجموعه پروتئینی‌های مرجع را اندازه‌گیری می‌کند.
\begin{equation}
Recall = \frac{N_{cb}}{|B|}
\end{equation}

\subsection{امتیاز F}
معیار امتیاز F میانگین همساز\footnote{\lr{Harmonic mean}} بین دو معیار دقت و بازیابی می‌باشد که به صورت مقابل محاسبه می‌شود:

\begin{equation}
F-score = \frac{2 \times Precision \times Recall}{Precision + Recall}
\end{equation}

\subsection{صحت}
معیار صحت به کمک دو معیار دیگر حساسیت خوشه‌بندی\footnote{\lr{Clustering-wise sensitivity (Sn)}}  و ارزش پیش‌بینی مثبت خوشه‌بندی\footnote{\lr{Clustering-wise positive predictive value (PPV)}} محاسبه می‌شود. با در نظر گرفتن $T_{i,j}$ به عنوان تعداد پروتئین‌هایی که هم در مجموعه پروتئینی $i$ ام و هم در مجموعه پروتئینی پیش‌بینی $j$ ام یافت می‌شوند و همچنین $N$ به عنوان تعداد پروتئین‌های مجموعه پروتئینی مرجع $i$، می‌توانیم $Sn$ و $PPV$ را به صورت مقابل تعریف کنیم:
$$
PPV = \frac{\sum_{j=1}^{|P|} max_{i=1}^{|B|}|T_{ij}|}{\sum_{j=1}^{|P|} \sum_{i=1}^{|B|} T_{ij}}
$$
$$
Sn = \frac{\sum_{i=1}^{|B|} max_{j=1}^{|P|}|T_{ij}|}{\sum_{i=1}^{|B|} N_i}
$$

از نظر مفهومی، معیار $PPV$ نشان‌دهنده نسبت مجموع بیشینه پروتئین‌های تطبیق‌یافته هر مجموعه پروتئینی پیش‌بینی‌شده با مجموعه‌های پروتئینی مرجع، به تعداد کل پروتئین‌های تطبیق‌یافته در مجموعه‌های پروتئینی پیش‌بینی‌شده است.
از سوی دیگر، معیار $Sn$ بیان‌کننده نسبت مجموع بیشینه پروتئین‌های تطبیق‌یافته هر مجموعه پروتئینی مرجع با مجموعه‌های پروتئینی پیش‌بینی‌شده، به تعداد کل پروتئین‌های موجود در مجموعه‌های پروتئینی مرجع است.
در نهایت به کمک این دو معیار می‌توان معیار صحت را به صورت مقابل محاسبه نمود:
$$
Acc = \sqrt{Sn. PPV}
$$