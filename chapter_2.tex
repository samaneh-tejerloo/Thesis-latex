\chapter{مفاهیم بنیادی}

\section{مقدمه}
در این فصل، مفاهیم بنیادی و پیش‌نیازهایی که برای درک بهتر پژوهش حاضر ضروری هستند معرفی می‌شوند. از آنجا که این پایان‌نامه در تقاطع علوم زیستی و روش‌های محاسباتی قرار دارد، آشنایی با مفاهیم هر دو حوزه برای دنبال‌کردن مطالب فصل‌های بعدی اهمیت ویژه‌ای دارد.
بر همین اساس، در بخش نخست این فصل، مفاهیم زیستی مرتبط با موضوع پژوهش مورد بررسی قرار می‌گیرند. سپس در بخش دوم، به معرفی مفاهیم محاسباتی مورد استفاده پرداخته می‌شود و تمرکز اصلی بر مباحث مرتبط با شبکه‌های عصبی گرافی و چارچوب‌های یادگیری مبتنی بر گراف خواهد بود.
در نهایت، در بخش پایانی این فصل، معیارهای ارزیابی به‌کاررفته در این پژوهش برای سنجش کیفیت شناسایی ماژول‌های عملکردی معرفی شده و به‌طور خلاصه تشریح می‌شوند تا زمینه لازم برای تحلیل نتایج در فصل‌های بعدی فراهم شود.

\section{مفاهیم زیستی}
مقدمه کوچک
\subsection{پروتئین}
\subsection{ماژول‌های عملکردی}
\subsection{شبکه‌های PPI}
\subsection{هستی شناسی ژن}
\subsection{بیان ژن}


\section{مفاهیم محاسباتی}

در مفاهیم بنیادی ابتدا به فرمول بندی مسئله خوشه‌بندی گراف‌های با گره‌های دارای ویژگی می‌پردازیم. 

\subsection{خوشه بندی در گراف های با گره های دارای ویژگی }
با فرض گراف $G=(V,E,F)$ که در آن $V$ مجموعه گره ها، $E$ مجموعه یال ها است و $F$ ماتریس ویژگی های گره ها می باشد، یک خوشه بندی از گراف $G$ را می توان با $C$ نشان داد که مجموعه ای از زیر مجموعه های $V$ است، به صورتی که  $C_i\in C\:;\:C_i\subset V$. هدف از خوشه بندی این است که خوشه هایی که هم از نظر ساختاری و هم از نظر ویژگی های گره‌ها بهم بیشترین شباهت را دارند، پیدا کنیم. همچنین خوشه‌های ایجاد شده باید از نظر ارتباط‌ یال‌های داخل خوشه چگال و در ارتباط یال‌ها با دیگر خوشه‌ها تنک باشند.

\subsection{دسته‌بندی و روش‌های کلی خوشه‌بندی گراف}
 روش‌های خوشه‌بندی گراف را می‌توان از دیدگاه‌های مختلفی تقسیم‌بندی کرد. این تقسیم‌بندی‌ها بر اساس معیارها و ویژگی‌های خاصی صورت می‌گیرند که به نحوه برخورد با داده‌های گرافی، نوع اطلاعات استفاده شده، و تکنیک‌های به کار گرفته شده بستگی دارد. در این پژوهش از آنجایی که نوع گراف ورودی مشخص است و قصد خوشه‌بندی گراف‌های \lr{PPI} با گره‌های دارای ویژگی را داریم، روش‌های خوشه‌ بندی را بر اساس روش‌ مورد استفاده تقسیم‌بندی می‌کنیم:

\begin{itemize}
    \item روش‌های طیفی\footnote{\lr{Spectral clustering}} : از مقادیر ویژه\footnote{\lr{Eigenvalues }}  ماتریس لاپلاسین یا مجاورت برای یافتن خوشه‌ها استفاده می‌کنند.
    \item روش‌های فاکتورگیری ماتریسی\footnote{\lr{Matrix factorization}} : از روش‌های تجزیه‌ ماتریسی مانند تجزیه نامنفی ماتریس\footnote{\lr{Non-negative matrix factorization}}  یا تجزیه مقدار تکین\footnote{\lr{Singular value factorization}}  برای ایجاد امبدینگ و خوشه‌بندی استفاده می‌کنند.
    \item روش‌های سلسله‌مراتبی\footnote{\lr{Hierarchical clustering}} : گراف را به صورت سلسله ‌مراتبی خوشه‌بندی می‌کنند که به دو روش تقسیمی و تجمعی دسته‌بندی می‌شوند.
    \item روش‌های مبتنی بر امبدینگ\footnote{\lr{Embedding-based methods}} : ابتدا گره‌ها به فضای برداری کم‌بعد نگاشت می‌شوند و سپس خوشه‌بندی روی این فضای برداری انجام می‌شود و تمرکز اصلی در این روش‌ها یافتن بازنمایی مناسب برای خوشه‌بندی گراف است. \lr{(Node2Vec, DeepWalk, GCN, GNN)}.
    \item روش‌های بدون امبدینگ\footnote{\lr{Non-embedding methods}} : مستقیماً از ساختار گراف برای خوشه‌بندی استفاده می‌شود بدون اینکه گره‌ها به فضای برداری منتقل شوند \lr{(Louvain, graph - cut based)}.

\end{itemize}

\subsection{پایگاه داده هستی شناسی ژن}
\lr{GO} یک بانک داده و سیستم طبقه‌بندی است که با هدف ایجاد یک زبان استاندارد برای توصیف ژن‌ها و محصولات ژنی (که پروتئین‌ها نیز جزو آنها هستند) ایجاد شده است. این سیستم شامل سه بخش اصلی است که هر یک از آنها یک جنبه خاصی از عملکرد زیستی را توصیف می‌کنند:

فرآیند زیستی\footnote{\lr{Biological process}} : این بخش به فرآیند‌های زیستی اشاره دارد که ژن و یا پروتئین خاصی در آن نقش دارد.

عملکرد مولکولی\footnote{\lr{Molecular function}} : این بخش عملکرد دقیق مولکولی ژن یا پروتئین را توصیف می‌کند.

مولفه‌ی سلولی\footnote{\lr{Cellular component}} : این بخش به مکانی که ژن یا پروتئین در آن قرار دارد اشاره می‌کند.
از ویژگی‌های دیگر این بانک داده نمایش اطلاعات به صورت سازماندهی شده و سلسله مراتبی است که شامل شبکه‌های بدون دور می‌شود و ویژگی‌ها به این صورت مرتب شده‌اند \cite{gene}. 

\subsection{شبکه‌های PPI و ویژگی‌های آنها }
یک شبکه \lr{PPI} معمولا به صورت یک گراف بدون جهت $G=(V,E)$ نشان داده می شود که $V$ و $E$ به ترتیب نمایان‌گر پروتئین ها و تعاملات بین آنها می باشند. وزن های روی یال ها را می توان برای توصیف ویژگی های شبکه \lr{PPI}، مانند ویژگی‌های توپولوژیکی یا عملکردی استفاده کرد. شبکه های \lr{PPI} سه ویژگی توپولوژیکی زیر را دارند:

\begin{itemize}
    \item توزیع بدون مقیاس\footnote{\lr{Scale-free distribution}} : $P(k)$   مفهوم توزیع درجه یعنی احتمال اینکه یک گره در یک شبکه دقیقا \lr{k} پیوند داشته باشد را نشان می دهد. یک  شبکه \lr{PPI} دارای توزیع درجه توانی $P(k)\sim\:k^{-\lambda}$ می باشد\cite{evolution} . این ویژگی به این معنی است که پروتئین‌های تعامل‌دار در شبکه‌های \lr{PPI} به طور یکنواخت توزیع نمی شوند‌، بیشتر پروتئین‌ها تنها در چند تعامل شرکت می‌کنند در حالی که مجموعه کوچکی از پروتئین‌ها در ده‌ها تعامل (تشکیل گره هاب\footnote{\lr{Hub}}) شرکت می‌کنند. 
    \item ویژگی جهان کوچک\footnote{\lr{Small-world property}} : پروتئین‌های یک شبکه \lr{PPI} دارای میانگین طول مسیر کم و ضرایب خوشه‌ای بالا هستند\cite{protein}  که سیگنال‌های هر گره در شبکه \lr{PPI} را قادر می‌سازد تا از طریق چند جهش به سرعت به هر گره دیگری برسند. در نتیجه شبکه‌های \lr{PPI} هم زمان انتقال سیگنال و هم زمان پاسخ کوتاهی خواهند داشت.
    \item شبکه با ماژول‌های عملکردی\footnote{\lr{Functional modular network}} : شبکه \lr{PPI} یک شبکه ماژولار و سلسله مراتبی می‌باشد. یک ماژول عملکردی در یک شبکه \lr{PPI} یک مجموعه با بیشترین تعداد پروتئین که عملکرد یکسانی دارند، می‌باشد. بارزترین مشخصه ماژول عملکردی، ارتباط بین ساختار توپولوژیکی شبکه \lr{PPI} و عملکرد‌ پروتئین‌های آن است که مبنای بسیاری از روش‌های تشخیص ماژول عملکردی است\cite{molecular} \cite{road}.
\end{itemize}

