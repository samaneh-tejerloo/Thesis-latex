بیوانفورماتیک یک حوزه میان‌رشته‌ای است که با استفاده از علوم زیست‌شناسی، کامپیوتر، ریاضیات و آمار به ذخیره‌سازی و تحلیل داده‌های زیستی می‌پردازد. با پایان‌یافتن پروژه توالی‌یابی ژنوم انسان و ورود به دوره‌ی پساژنی، تحقیقات پروتئومیک به یکی از مهم‌ترین حوزه‌های علوم زیستی تبدیل شده است. پروتئومیک به مطالعه ویژگی‌های پروتئین‌ها برای توصیف ساختار، عملکرد و کنترل سیستم‌های زیستی می‌پردازد. پروتئین‌ها اغلب به تنهایی عمل نمی‌کنند، بلکه با هم تعامل دارند و برای انجام وظایف زیستی، به مولکول‌های بزرگ‌تری تبدیل می‌شوند. تعاملات بین پروتئین‌ها را به کمک ساختار شبکه‌ای به نام شبکه تعامل پروتئین‌-‌پروتئین نمایش می‌دهند. یک ترکیب پروتئینی در شبکه‌های PPI یک ساختار مولکولی است که هم از نظر ویژگی و هم از نظر ساختاری از پروتئین‌های سازگار با هم تشکیل شده است. با تحلیل شبکه‌های \lr{PPI} می‌توانیم این مجموعه از پروتئين‌ها را شناسایی کنیم. یکی از مسائل مهم در بیوانفورماتیک کشف ماژول‌های پروتئین در شبکه‌های تعامل پروتئین‌-‌پروتئین است. کشف این ماژول‌ها معادل مسئله‌ی کشف انجمن در گراف‌ است. در بسیاری از کاربردهای بیوانفورماتیکی کشف ماژول‌های پروتئینی با استفاده از الگوریتم‌های کشف انجمن در گراف انجام می‌شود. در این پژوهش ما قصد داریم روشی ویژه برای کشف انجمن در شبکه‌های تعامل پروتئینی طراحی کنیم که علاوه بر در نظر گرفتن ساختار گرافی برای شناسایی ماژول‌ها به ویژگی‌های زیستی پروتئین‌ها نیز توجه دارد.  برای مثال، استفاده از اطلاعات زیستی پروتئین‌ها که در پایگاه‌های داده‌ای مانند \lr{GO} و \lr{KEGG} ذخیره شده‌اند، همراه با داده‌های بیان ژنی و ترکیب این اطلاعات با شبکه \lr{PPI} می‌تواند به شناسایی دقیق‌تر و کارآمدتر ماژول‌های پروتئینی کمک کند. از این روی، در این پژوهش ما قصد معرفی یک الگوریتم خوشه‌بندی برای شبکه‌های \lr{PPI} بر پایه شبکه‌های عصبی گرافی و با در نظر گرفتن ویژگی‌های گره‌‌ها داریم.    
\keywords{شبکه‌های عصبی گرافی، تعامل پروتئین-پروتئین، شناسایی ماژول‌های عملکردی، خوشه‌بندی گراف‌های دارای ویژگی}